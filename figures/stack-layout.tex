\begin{tikzpicture}[scale=.8]

  % Segments
  \drawstruct{(-10,0)}
  \structcell[freecell]{\code{size}}         \coordinate (seg_start) at (currentcell.east);
  \structcell[freecell]{\code{length}}
  \structcell[freecell]{\code{content}}      \coordinate (seg_content) at (currentcell.east);
  \structcell[freecell]{\code{head}}         \coordinate (seg_head) at (currentcell.east);
  \structcell[freecell]{\code{data\_cursor}} \coordinate (seg_cursor) at (currentcell.east);
  \structcell[freecell]{\code{next}}         \coordinate (seg_next) at (currentcell.east);
  \structcell[freecell]{\code{prev}}
  \structname{Stack segment}

  \drawstruct{(-10,-10)}
  \structcell[freecell]{\code{size}}         \coordinate (seg1_start) at (currentcell.east);
  \structcell[freecell]{\code{...}}
  \structcell[freecell]{\code{prev}}         \coordinate (seg1_prev) at (currentcell.east);
  \structname{Stack segment}

  % Segment content
  \startframe{}
  \cell{\code{declared\_type}} \coordinate (stack_el1_e) at (2,\value{cellnb});
                               \coordinate (stack_el1_w) at (-2,\value{cellnb});
  \cell{\code{actual\_type}}
  \cell{\code{next}}
  \padding{3}{\code{data}}
  \finishframe{Stack element}


  \startframe{}
  \cell{\code{declared\_type}} \coordinate (stack_el0_start) at (-2,\value{cellnb});
  \cell{\code{actual\_type}}
  \cell{\code{next}}           \coordinate (stack_el0_e) at (2,\value{cellnb});
  \padding{1}{\code{data}}
  \finishframe{Stack element}

  \cell{...} \coordinate (stack_cursor) at (-2,\value{cellnb});

  % element next ptr
  \draw[*->, thick, bend right=20] (stack_el0_e) to (stack_el1_e);
  % segment content ptr
  \draw[*->, thick] (seg_content) to (stack_el1_w);
  % segment cursor ptr
  \draw[*->, thick] (seg_cursor) to (stack_cursor);
  % segment head ptr
  \draw[*->, thick] (seg_head) to (stack_el0_start);
  % segment ll
  \draw[*->, thick] (seg_next) to (seg1_start);
  \draw[*->, thick, bend right=30] (seg1_prev) to (seg_start);

\end{tikzpicture}


% \begin{drawstack}
%   \startframe
%   %   \cellcom writes something on the right-hand side of a cell.
%   \cell{loc2} \cellcom{-8(\%ebp)}
%   \cell{loc1} \cellcom{-4(\%ebp)}
%   %   \esp and \ebp are stack pointer and base pointer in Pentium.
%   %   These macros are simple shortcuts for \cellptr{...}
%   \cell{Sauvegarde \%ebp} \esp \ebp
%   \cell{@ retour} \cellcom{4(\%ebp)}
%   \finishframe{fonction\\ {\tt f}}
%   \startframe
%   \cell{} \cellcom{8(\%ebp)}
%   \cell{}
%   \finishframe{fonction\\ {\tt main}}
% \end{drawstack}

% \section{Padding}

% \begin{drawstack}
%   \cell{above padding}
%   \padding{3}{nothing here}
%   \cell{below padding}
% \end{drawstack}

% \section{Below/Above stack pointer}

% \begin{drawstack}
%   \cell{Top}
%   \cell{Below top}
%   %   \bcell is just like \cell, but in a different color.
%   \bcell{Above bottom} \cellptr{Stack pointer here}
%   \bcell{Bottom}
% \end{drawstack}

% \section{Highlighting some cell}

% \begin{drawstack}
%   \cell{Uninteresting cell}
%   \cell{Interesting cell} \cellround{Yes, this one!}
% \end{drawstack}

% \section{Structures without a stack structure}

% \begin{tikzpicture}
%   \draw (3, -1) node (Otm) {
%   \begin{tabular}{c}
  %       Object\\vtable
  %     \end{tabular}
  %       };

  %       \drawstruct{(0,0)}
  %       \structcell[freecell]{~} \coordinate (Atm) at (currentcell.east);
  %       \structcell[freecell]{\texttt{@Object.equals()}}
  %       \structcell[freecell]{\texttt{@code A.m()}}
  %       \structcell[freecell]{\texttt{@code A.p()}} \coordinate (A) at (currentcell.west);
  %       \structname{
  %       \begin{tabular}{c}
  %       A's vtable
  %     \end{tabular}
  %       }

  %       \drawstruct{(-4,-3)}
  %       \structcell[freecell]{} \coordinate (Btm) at (currentcell.east);
  %       \structcell[freecell]{\texttt{@Object.equals()}}
  %       \structcell[freecell]{\texttt{@code A.m()}}
  %       \structcell[freecell]{\texttt{@code B.p()}}
  %       \structcell[freecell]{\texttt{@code B.q()}}
  %       \structname{B's vtable}

  %       \draw[->] (Btm) -- (A);
  %       \draw[->] (Atm) -- (Otm);
  %       \end{tikzpicture}

  %       \section{Structures and stack together}

  %       \begin{tikzpicture}[scale=.8]

  %       \stacktop{}
  %       \separator
  %       \cell{\texttt{p3}}        \cellcomL{11(GB)} \coordinate (p3) at (currentcell.east);
  %       \separator
  %       \cell{\texttt{p2}}        \cellcomL{10(GB)} \coordinate (p2) at (currentcell.east);
  %       \separator
  %       \cell{\texttt{p1}}        \cellcomL{ 9(GB)} \coordinate (p1) at (currentcell.east);
  %       \separator
  %       \cell{\texttt{@P3D.diag}} \cellcomL{ 8(GB)}
  %       \cell{\texttt{\footnotesize @Object.equals}} \cellcomL{ 7(GB)}
  %       \cell{\texttt{3(GB)}}     \cellcomL{ 6(GB)} \coordinate (T1) at (currentcell.east);
  %       \separator
  %       \cell{\texttt{@P2D.diag}} \cellcomL{ 5(GB)}
  %       \cell{\texttt{\footnotesize @Object.equals}} \cellcomL{ 4(GB)}
  %       \cell{\texttt{1(GB)}}     \cellcomL{ 3(GB)} \coordinate (T2) at (currentcell.east);
  %       \separator
  %       \cell{\texttt{\footnotesize @Object.equals}} \cellcomL{ 2(GB)}
  %       \cell{\texttt{null}}      \cellcomL{ 1(GB)}
  %       \cell[draw=none]{Stack}


  %       \drawstruct{(5,1)})
  %       \structcell{z=2,5}
  %       \structcell{y=2,5}
  %       \structcell{x=2,5}
  %       \structcell{.} \coordinate (O1) at (currentcell.west);
  %       \coordinate (O1l) at (currentcell.south);

  %       \drawstruct{(9,-3)}
  %       \structcell{y=1}
  %       \structcell{x=1}
  %       \structcell{.} \coordinate (O2) at (currentcell.west);
  %       \coordinate (O2l) at (currentcell.south);

  %       \draw[->] (p3) -- (O1);
  %       \draw[->] (p2) -- (O1);
  %       \draw[->] (p1) -- (O2);

  %       \draw[->] (O1l) .. controls (O1 |- T1) .. (T1);
  %       \draw[->] (O2l) .. controls (O2 |- T2) .. (T2);

  %       \draw (10,-10) node{Heap};

  %       \end{tikzpicture}



  %       \section{Using tikzpicture instead of drawstack}

  % %       The environment drawstack is basically a syntactic sugar for
  % %
  % %       \begin{tikzpicture}[#1]
  % %         \stacktop{}
  % %         ...
  % %         \stackbottom
  % %       \end{tikzpicture}
  % %
  % %       You can use the above syntax for more flexibility.

  %       \begin{tikzpicture}[scale=0.8]
  %       \small
  %       \stacktop{}
  %       \cell{My cell}
  %       \stackbottom{}
  %       \end{tikzpicture}

  %       \section{Changing style}

  %       {% tikzstyle will be local to this {...}
  %       \tikzstyle{freecell}=[fill=blue!10,draw=blue!30!black]
  %       \tikzstyle{occupiedcell}=[fill=blue!10!orange!10,draw=blue!30!black]
  %       \tikzstyle{padding}=[fill=yellow!20,draw=blue!30!black]
  %       \tikzstyle{highlight}=[draw=orange!50!black,text=orange!50!black]

  %       \begin{drawstack}
  %       \cell{Uninteresting cell}
  %       \cell{Interesting cell} \cellround{Yes, this one!}
  %       \bcell{bcell}
  %       \padding{2}{Padding}
  %       \end{drawstack}
  %       }

  %       \section{Example: Computing Factorial}

  %       \begin{drawstack}[scale=0.8]
  %       \startframe
  %       \cell{N = 1}
  %       \cell{...}
  %       \finishframe{fact(1)}
  %       \startframe
  %       \cell{N = 2}
  %       \cell{...}
  %       \finishframe{fact(2)}
  %       \cell{$\vdots$}
  %       \startframe
  %       \cell{N = 5}
  %       \cell{...}
  %       \finishframe{fact(5)}
  %       \end{drawstack}

%%% Local Variables:
%%% mode: latex
%%% TeX-master: "../report"
%%% End:
