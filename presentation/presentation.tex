\documentclass[handout]{beamer}

\usepackage[utf8]{inputenc}
\usepackage{tikz}
\usepackage{drawstack}
\usepackage{listings}

\usetikzlibrary{shapes.multipart,arrows}

% Setup
\title{Abstract Machine~\\for Modern Programming Languages}
\author[Altschuler, Færevaag]{Simon Altschuler (s123563) \and Markus Færevaag (s123692)}
\date{Software Technology Bachelor Thesis, Spring 2015}
\subject{Computer Science}

% Style
\usetheme{Berlin}
\usecolortheme{beaver}
\usefonttheme{structuresmallcapsserif}
\beamertemplatenavigationsymbolsempty{}
\setbeamertemplate{footline}[frame number]
\setbeamertemplate{itemize item}[circle]
\setbeamertemplate{itemize subitem}[triangle]
\setbeamertemplate{enumerate items}[default]
\definecolor{lered}{rgb}{0.58, 0.02, 0.03}
\setbeamercolor{itemize item}{fg=lered}
\setbeamercolor{itemize subitem}{fg=lered}

\newcommand{\n}[1]{\\~\texttt{\color{red}\tiny #1}}
% \newcommand{\n}[1]{}

\begin{document}

\frame{\titlepage}

\begin{frame}
  \frametitle{Motivation}
  \framesubtitle{Reasons for working on this project}

  Problem
  \begin{itemize}
  \item Available machines are unsuitable for modern paradigms
    \n{Not supporting dynamic and functional languages}
  \item Heavy focus on object-orientated and imperative languages
  \end{itemize}

  \pause{}

  \vspace{20pt}
  Learning
  \begin{itemize}
  \item Abstract machines are used everywhere
    \n{Browsers, mobile devices, desktop apps, embedded systems, similar to compilers/interpreters}
  \item Embody a broad spectrum of computer science topics
    \n{Compiler theory, lots of data structures, optimizations, low-level stuff}
  \end{itemize}

\end{frame}

\begin{frame}
  \frametitle{Contributions}
  \framesubtitle{Division of Responsibility}

  \begin{columns}[onlytextwidth, t]
    \begin{column}{0.33\textwidth}
      Shared
      \fontsize{9pt}{15}\selectfont
      \begin{itemize}
      \item Object model
      \item Stack
      \item Core architecture
      \item Testing
      \item Debugging facilities
      \end{itemize}
    \end{column}

    \pause{}

    \begin{column}{0.33\textwidth}
      Simon
      \fontsize{9pt}{15}\selectfont
      \begin{itemize}
      \item Instruction cycle
      \item Execution model
      \item ELF
      \end{itemize}
    \end{column}

    \pause{}

    \begin{column}{0.33\textwidth}
      Markus
      \fontsize{9pt}{15}\selectfont
      \begin{itemize}
      \item Type system
      \item Threads
      \item Information tables
      \end{itemize}
    \end{column}
  \end{columns}

  \pause{}

  \vspace{30pt}
  \centering
  We have both been involved in everything!

\end{frame}

\begin{frame}
  \frametitle{Agenda}
  \fontsize{11pt}{20}\selectfont
  \begin{enumerate}[<+->]
  \item Abstract machines 101
    \n{Simon}
  \item Existing systems
    \n{Markus}
  \item Matisse and the multi-paradigm approach
    \n{Simon}
  \item Implementation
    \n{Markus}
  \item Demo
    \n{Simon}
  \item Evaluation
    \n{Markus}
  \item Conclusions
    \n{Foo bar}
  \end{enumerate}
\end{frame}

\begin{frame}
  \frametitle{Abstract machines 101}

  Fundamentals

  \begin{itemize}[<+->]
  \item Virtual computer architecture
    \n{PROCESS not SYSTEM! Computer arch: organization of components}
  \item Hardware implemented in software
    \n{Shares properties of hardware like CPU/Memory mgmt/clock cycles}
  \item Executes low-level machine code
    \n{Machine code vary from very-low level to relatively high-level}
  \end{itemize}

\end{frame}

\begin{frame}
  \frametitle{Abstract machines 101}

  Advantages
  \begin{itemize}[<+->]
  \item Platform independence
    \n{Unified ISA, Just compile for new platforms}
  \item Expressive instruction set
    \n{Main reason, high-level constructs}
  \item Security
    \n{Sandboxed environment, cannot control access to system resources}
  \end{itemize}

  Disadvantages
  \begin{itemize}[<+->]
  \item Performance overhead
    \n{JIT = GG, but still some compilation overhead}
  \item Users must have the machine installed
    \n{Must be retrieved from somewhere, it must be available for the system}
  \end{itemize}

\end{frame}

\begin{frame}
  \frametitle{Abstract machines 101}

  How and where are they used
  \begin{itemize}[<+->]
  \item Web browsers
    \n{Java applet}
  \item Desktop applications
    \n{JVM, CIL, Parrot}
  \item Embedded devices
    \n{Hmm?}
  \item Virtualization
    \n{Hmm?}
  \end{itemize}

\end{frame}

\begin{frame}
  \frametitle{Existing systems}
  \framesubtitle{And their shortcomings}

  \begin{itemize}
  \item Two most popular/widespread/sick: JVM and CIL
  \item Why they are insufficient for modern paradigms like seriously, uKnow
  \item Interesting case: Parrot
  \end{itemize}
\end{frame}

\begin{frame}
  \frametitle{Matisse}
  \framesubtitle{The Multi-Paradigm Approach}

  What distinguishes Matisse
  \begin{itemize}
  \item No paradigm assumptions
    \n{Attempted to be flexible enough to cover all}
  \item Achieved by exposing primitives
    \n{Building blocks for compilers, to use however they like. Scopes, fields, types, sub-routines}
  \item Opt-in features
    \n{Threading, scopes}
  \item Sub-routines are simple and sharable
    \n{They are just references and anon. could be generated easily}
  \item Flexible type system
    \n{AnyType is awesome for dynamic}
  \end{itemize}
\end{frame}

\begin{frame}
  \frametitle{Implementation}

  \begin{itemize}
  \item Note: Reasons for using C
  \item Overview diagram
  \item Explanation of steps taken from loading a file to executing an
    instruction to shutting down
  \end{itemize}
\end{frame}

\begin{frame}
  \frametitle{Demo}
  \begin{itemize}
  \item Program 1
    \begin{itemize}
    \item Stack manipulation
    \item Sub-routines
    \item Arithmetic
    \end{itemize}

  \item Program 2
    \begin{itemize}
    \item Heap object
    \item Field manipulation
    \end{itemize}

  \item Complex program
  \end{itemize}
    % \item Demo program 1:
    %   \begin{itemize}
    %   \item push two ints
    %   \item add
    %   \item show stack
    %   \item make one int a float
    %   \item type error!
    %   \item box
    %   \item make sub-routine that takes the boxed int and adds one (and returns nothing)
    %   \item unbox the reference
    %   \item use `iprint' to print the result (strtab 4 is iprint)
    %   \end{itemize}

    % \item Demo program 2:
    %   \begin{itemize}
    %   \item new heap object (new, `person')
    %   \item push field `age' (should be 0)
    %   \item push an int
    %   \item pop to field `age'
    %   \end{itemize}

    % \item Show the fibo program as an example of a more complex thing
\end{frame}

\begin{frame}
  \frametitle{Performance Evaluation}

  WE DIDN'T TURN OFF JIT IN MONO

  \begin{itemize}
  \item Show and explain fibo graph
  \item Show and explain heap graph
  \item Show and explain heap profile (try running it again to get an exponential version)
  \end{itemize}
\end{frame}

\begin{frame}
  \frametitle{Conclusions}

  \begin{itemize}
  \item Did we succeed?
    \n{A good proof-of-concept, basics work, performance is bad (expected)}
  \item What we learned
    \n{AM internals, primitives needed for language impl., C design and architecture}
  \item Future of Matisse
    \n{More implementation}
  \end{itemize}
\end{frame}

\end{document}