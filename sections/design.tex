In the following we will outline  design of \thename{} and give a detailed
description of each area of functionality. The design is heavily tied to the
\thename{} specification, but further includes some considerations for
implementation, i.e.~the design of the implementation.

As previously mentioned, the primary aim of \thename{} is to provide an abstract
machine that is capable of supporting most of the modern programming paradigms
in an efficient manner. The overarching idea is that by exposing much of the
underlying low-level constructs we provide very flexible means for compilers to
do things the way that is most suitable for their particular requirements,
without having to conform to a tight object-model or strict type system. This
involves exposing constructs like scopes, object model internals (field and
member management and modification), thread management and type system. Further
the type system is aimed at being strict enough to guarantee type safety as far
as possible, but flexible enough for dynamic languages. Compilers are free to
select which features to utilize, meaning for instance that for a strictly
statically typed language, the dynamic typing mechanisms can be disregarded and
all method dispatches can be wired statically.
% TODO more?

\textbf{Note:} Because the specification is such a central document to the
design we will not cite it directly throughout the thesis. Instead we refer to
Appendix~\ref{appendix:spec} where it is included in its entirety.

\subsection{Memory and Object Model}

\thename{} uses the common stack and heap memory model. Each running thread has
their own evaluation stack that is local to itself. Other threads, whether
parent or child can not access elements on another thread's stack, neither by
direct memory addressing (TODO: maybe with unsafe code?) or by use of the
\code{StackReference} type. The only means by which a thread can make data from
its stack available to others is by boxing it, i.e.~moving it to the heap and
track it by reference. The heap is a single, usually large, globally shared
memory region where data is stored and modified by use of references. It is
managed fully by the garbage collector, which means that all allocations and
deallocations are made through an interface. Internally the garbage collector is
regarded as a ``black box'', meaning that no assumptions to its inner workings
are made, and it is assumed to efficiently free unused memory when possible.

The object model defines how objects are layed out in heap and how their content
is described by the type system. \thename{} uses a simple but very flexible
object model. The general idea is that a heap object is described by the
\code{Composite} type which is capable of expressing arbitrarily complex data
structures. In addition to what is described by the type a heap object includes
a reference to an object of the \code{AnyType} type. The purpose of this is to
provide language implementations the means to associate some arbitrary
information to objects. It could used for any kind of meta data storage,
implementation of class methods, polymorphism, etc.

%TODO the figure is outdated. again.
\begin{figure}[H]
  \centering
  \begin{tikzpicture}
  \draw (9, -1) node (b_vtbl) {
    \begin{tabular}{c}
      Bootstrapped\\vtable
    \end{tabular}
  };

  \drawstruct{(5,0)}
  \structcell[freecell,fill=blue!10]{Virtual table} \coordinate (vtbl_e) at (currentcell.east); \coordinate (vtbl_w) at (currentcell.west);
  \structcell[freecell]{RTTI}
  \structcell[freecell]{Function 1}
  \structcell[freecell]{Function 1}
  \structname{Virtual table}

  \drawstruct{(5,-5)}
  \structcell[freecell]{KV1} \coordinate (hm) at (currentcell.west);
  \structcell[freecell]{KV2}
  \structname{Hash map}

  \drawstruct{(0,0)}
  \structcell[freecell,fill=blue!10]{Virtual table} \coordinate (ho_vtbl) at (currentcell.east);
  \structcell[freecell,fill=blue!10]{Hash map} \coordinate (ho_hm) at (currentcell.east);
  \structcell[freecell]{Field 1}
  \structcell[freecell]{Field 2}
  \structcell[freecell]{Simple function 1}
  \structcell[freecell]{Simple function 2}
  \structname{Heap object}

  \draw[->, thick] (ho_vtbl) -- (vtbl_w);
  \draw[->, thick] (ho_hm) -- (hm);
  \draw[->, thick] (vtbl_e) -- (b_vtbl);
\end{tikzpicture}
  \caption{Heap object memory layout}
  \label{fig:design:heap-object-layout}
\end{figure}

\subsection{Execution Model}
%TODO: tasks and transactions
The model of execution defines how, where and by whom code is executed during
run-time. All code in \thename{} is executed in the context of a thread (details
in Section\ref{sec:design:threading}), and all threads are spawned from the
initial ``main'' thread in which execution begins. This allows language
implementations to use the multi-threading features of \thename{}, but also to
ignore them and run everyting in a single thread without having to worry about
creating and destroying threads.

Most CPU architectures define a \emph{calling convention} which defines how
parameters are passed to sub-routines and how values are returned from
them. Parameters to a subroutine must be passed into the stack prior to an
\code{invoke} instruction. The sub-routine code can then peek upward into the
stack to retrieve or replace them. \thename{} has no notion of return values as
commonly used by other architectures. Rather, pass-by-reference parameters are
used to pass values back to the caller. The callee allocates space on the stack
or the heap and sub-routines must accept corresponding reference type parameters
which it can then ``fill'' with the values that are to be returned. This is a
flexible method that allows any number of values to be passed back to the
caller, and further does not require any dedicated instructions for returning
values.

Another aspect of the execution model is the use of scopes. At its core, a scope
is simply a mapping from names to values that define the bindings available at
any given point during execution. The available bindings depend on the scoping
rules used, and can generally be divided into dynamic and lexical
mechanism. With dynamic scoping a name is resolved when it is \emph{evaluated},
that is, it depends on the execution context or state of the program. With
lexical scoping a name is resolved when it is \emph{defined}, meaning that
resolution depends on the name's lexical location in the source code (hence the
name). Dynamic scoping can be confusing and make it difficult to reason about a
program's behavior. It can lead to unwanted or unexpected results because the
programmer has to be very careful not to use identical names in different
locations that might be evaluated in the same execution context. Lexical scoping
makes it easier to figure out what a program will do and harder to shoot
yourself in the foot. Dynamic scoping has more or less disappeared from language
implementation\cite{cse341}.

Scopes are usually organized in tree structures, where each scope has exactly
one parent and zero or more children. Name resolution starts in the current
scope, and if a matching name is found then the value of the name is determined
immediately. If the name was not found in the current scope then resolution
continues in the scope's parent, repeatedly. This means that a name can be
override, known as \textit{shadowing}, by a child scope by binding a value to a
name that is also used by the parent scope.

\thename{} exposes run-time functions for managing scopes, more specifically
pushing and popping them on demand. That means that by default a new scope is
\textit{not} introduced when a sub-routine is invoked, compilers must handle
this manually. This way a compiler is completely free to introduce scopes when
it is required by the language specification. For instance, some languages
introduce scopes in if-blocks (e.g. C) and some do not (e.g JavaScript), both of
which are seamlessly supported. There is little to no performance overhead by
manual scope management, because the internal procedures are exactly the same as
if the machine would introduce scopes automatically.

\subsubsection{Sub-routines}

%TODO

\subsection{Executable File Format}
%TODO: more
\thename{} reads executable files written in the Executable Linkable Format
(ELF). ELF is a very popular format for executable binary files that is used in
most Unix based systems (OpenBSD, Linux, Solaris, etc\cite{NEEDED}). That alone
enables much easier porting of existing libraries and frameworks to
\thename{}.

\subsection{Stack Management}
\label{sec:design:stack-mgmt}
%% Basic
%% Stack vs. Register
%%% (see background)
%% Threaded
%% Stack element
%%% Type
%% Manipulation (instr)

A stack is an abstract data type, fundamental in the fields of algorithms and
computer science. At its core, it is a very simple model consisting of two
essential functions, {\it push} and {\it pop}, as visualized in
figure~\ref{fig:stack}. Push adds an element to the top of a stack, while pop
removes the top most element. As an analogy, imagine a pile of dishes: when
adding a new dish it becomes the new top-most dish in the pile. This sequence of
adding and taking from a collection of element is called Last In, First Out
(LIFO).
\begin{figure}[h]
  \centering
  \includegraphics[scale=0.6]{images/stack.png}
  \caption{Stack push and pop operations}
  \label{fig:stack}
\end{figure}

% The stack was first introduced in 1946 by Alan Turing, while he was working on
% the Automatic Computing Engine (ACE)~\footnote{Automatic Computing Engine (ACE):
% \url{http://en.wikipedia.org/wiki/Automatic_Computing_Engine}} at the National
% Physical Laboratory (NPL)~\footnote{National Physical Laboratory:
% \url{http://en.wikipedia.org/wiki/National_Physical_Laboratory_(United_Kingdom)}}
% in the UK:
% \begin{quote}
%   [Alan Turing] outlined a method for leaving one line of work being carried out
%   on the computer, calling up a subsidiary line, and then returing to the first
%   line when finished with the subsidiary line. He called the calling up and
%   returning routines BURY and UNBURY.~(\textcite[82]{newton})
% \end{quote}
% Here, BURY and UNBURY are synonymous with push and pop.

% @book{newton,
% author    = "David E. Newton",
% year      = 2003,
% title     = "Alan Turing : a study in light and shadow",
% publisher = "[Philadelphia]: Xlibris",
% ISBN      = 9781401090791
% }

Although, essentially, one cannot take the second element from top without first
removing the top most element, the model can easily be modified to support such
operations. For instance, a common stack operation is {\it peek}, which reads
the value of the top element without actually removing it. This can easily be
done by popping, storing the value and pushing it back to the stack. Other
operations can be implemented in a similar fashion. %TODO: thats not how we actually do it

\subsubsection{Stack vs. Register Machine}
Machines can implement the stack model virtually and execute it directly in
hardware, but as we are designing an abstract machine, we will only discuss the
virtual aspect. Regardless, such a machine is commonly called a \term{stack
  machine}.

The most common notation in arithmetic formulas and statements are the infix
notation, where the operator is written between its operands: $1 + 2$. One can
express the same statement with prefix notation, also called \term{Polish
  notation}: $+\ 1\ 2$. When programming stack machines, one uses what is called
the \term{reverse Polish notation}, which is in fact postfix notation:
$1\ 2\ +$. By using this notation, one can easily convert this into stack
operations, where $1\ 2\ +$ becomes:
\begin{stackops}
  \op{push 1}{1}
  \op{push 2}{2,\ 1}
  \op{add}{3}
\end{stackops}

Here the {\tt add} operation pops two elements, adds them together and pushes
the result back onto the stack.

As briefly discussed in Section~\ref{sec:background:stack-vs-register}, a
register machine TODO

\subsubsection{Stack Organization}
% All stacks will initially have a single activation frame

Stacks in \thename{} are unbounded, meaning that they can be arbitrarily large
and that stack overflows never can occur (if the machine runs out of memory
entirely it is treated as a memory error, not a stack overflow). Throughout the
report, we will say that the stacks \term{grows downwards}. This is because
physically, the top most element of the stack is located at a low address in the
host computers memory. Therefore, as more elements are pushed to the stack, they
will get higher addresses, making the stack grow downwards in memory.

A stack element will consist of some type information along with the actual data
bytes. The type information consist of a declared type, but also an actual
type. For statically typed languages, these will most likely (TODO: ?) be the
same. For dynamic languages this is required to keep track of an element's type,
as it is allowed to assign a value type an element, which was originally
declared with an other type.

\subsubsection{Stack Manipulation}
As mentioned above, there will we instructions for doing simple stack
manipulations, for instance {\tt push} and {\tt pop}. As we will see later,
these two instructions alone will be very limiting, and make the job of writing
a compiler very tedious. Therefor, we will need more instructions, making more
complex manipulations more convenient.

When we later describe the execution model of the machine, we will see that we
need instructions for pushing an popping elements further up the stack. For
instance, if we want to access the arguments given to a subroutine, they will be
located up the stack. The compiler will therefor need an effective method of
duplicating that element onto the top of the stack, making is accessible for
further computation. Also it will need to push an element to a specific position
up the stack, effectively returning a value to the caller.

Such versions of the push and pop operations will take an displacement argument,
saying how many places up the stack it is to push to or pop from. To make this
as convenient as possible, the displacement will be from the top of the stack,
meaning displacement 0 will be the top of the stack. Exception handling will
have to be utilized when using a displacement which is not allowed, for instance
outside the stack or its own scope.

Lets call these two instructions {\tt pushElement} and {\tt
  storeElement}. Unlike {\tt push}, which adds a new element to the stack, the
specification defines {\tt pushElement} to overwrite the element at the given
displacement. This is more convenient as this does not change the displacement
of the following elements, though it requires some planning, as there will have
to be an element on the stack with the purpose of being overwritten. {\tt
  storeElement} will pop of the top-most element and place it at the given
displacement.

Also, unlike our previous stack example, the {\tt push} instruction, or rather
the {\tt pushElement} instruction, will not take a value as an argument. Rather
it will take the element at the given displacement and push onto the top of the
stack. There will be separate instructions for adding new elements onto the
stack, for instance {\tt pushConstant}.

Here we can see some simple operations using the {\tt pushElement} and {\tt
  popElement} operations.
\begin{stackops}
  \op{-}{3,\ 1,\ 1}
  \op{pushElement 0}{3,\ 1,\ 1,\ 3}
\end{stackops}


\subsection{Threading}
\label{sec:design:threading}
Big chip manufacturers like Intel, have always been under big pressure to
deliver ever faster processors, year after year. For around the last decade, the
technological development of processors has been focused adding cores rather
than increasing the clock speed, which already started to stagnate around
2004~\cite{sutter}. With multiple processors, or popularly called a multi-core
processor, processes may actually run in parallel, i.e. running at the same
time, rather than just concurrently where multiple processes actually share the
same processor.

A thread is typically the smallest unit of executable instructions by a
processor. A process being run by an operating system therefore had one thread
which the operating systems scheduler delegated CPU time to. With multi-core
processors, the notion of a single process having multiple threads arose, making

Although multi-threading is a very interesting feature, it also raises a lot of
challenges when designing machines. Different threads cannot directly read and
write to the same place, or address, in memory without taking data races into
account, which is a common mistake when writing multi-threaded programs. If
multiple threads are reading and writing to the same memory, it is impossible
for the threads to base any logic on it, as another thread can change its value
at any time, effectively pulling the rug from under you. This problem as been
named the mutual exclusion problem, and is solved by using synchronization
structures.

Our machine will both support threading and synchronization structures
therefor. As we are using a stack machine, each thread has to have its own
state, including its own stack. In other words, all stacks are private and
cannot be shared across threads. This can though be facilitated by using heap
object which are referenced from multiple threads' stack.

\subsubsection{Tasks and Transactions}
TODO
% Multiple executables may execute simultaneously.
% Task and transaction semantics is supported. A transaction is considered
% executed by the thread starting it. A spawned task is independent of the spawn-
% ing thread and may be executed by the spawning thread or some other thread.
% However, tasks will not migrate between threads so the thread which started
% execution of a task will also complete the task.

% Tasks do not share stacks with the spawning thread, spawning task, or any
% other task. Implementations may, however, execute tasks using the stack of the
% executing thread. The executed task is not allowed to access any stack space
% beyond its initial set of stack elements. The implementation must throw an
% exception if any such offending access is attempted.

\subsubsection{Thread Pool}
Creating and destroying system threads is not free. If a new thread is created
on demand, it would create significant overhead. Another problem with this model
is the risk of resource thrashing. If there is no limit on the number of threads
that can be created, it can intentionally or unintentionally be exposed to heavy
abuse. A solution to both of these challenges is having what is called a
\term{thread pool}. On machine initialization, a set number of threads is
pre-initialized and stored in a data structure. When a thread tries to spawn a
new thread, it is given one from the thread pool. If the thread pool is empty,
it has to wait for a thread to become vacant.

This model, although effective, also brings with it some
challenges. Specifically, when the thread pool is empty. The spawning thread
cannot stop and wait until a new thread is vacant, as this would block the whole
thread. Rather, the machine will use a work queue. A work queue manages the
thread pool so other threads does not need to wait for a thread to become
vacant, and thereafter fight with each other of whom gets to take it. The
spawning threads would give the task which it wants run in a new thread, to the
work queue. The queue would monitor the thread pool and as soon as a thread
becomes free, it would take the task first in the work queue and run in.

\subsubsection{Errors and Exceptions}
Two types of errors can occur; internal errors in the machine and errors
produced by the program the machine is executing. Internal errors will most
likely occur in some child thread and not in the process' original thread. If
that is the case we would stop that thread and gracefully stop the machine if it
cannot safely continue. If the error occurs in the process' original thread, we
will in most cases still be able to catch the signal sent by the OS and
gracefully stop.

When the program being executed produces an \term{exception} it will percolate
up through a defined set of exception handlers until it is handled. If there is
no way of handling the exception it called an uncaught exception. It that case
the machine will print some information on the exception and stop. It will not
attempt to join all threads (TODO) as it cannot know how long this will
take. Therefore all threads are exited and memory is freed, ensuring no memory
leaks.

TODO: Exception handling


\subsection{Types}

The type system of \thename{} aims to enable efficient implementation of both
dynamically and statically typed languages. That gives rise to a need for an
adaptable system that is capable of describing arbitrarily complex types along
with potential storage layout information for languages that require explicit
memory layout of objects. Moreover the system must be flexible and modifiable
during run-time, e.g.~to allow dynamic languages to add and remove members of an
object while still being able to properly type check the code.

\subsubsection{DWARF}

Types are expressed in the DWARF format\cite{dwarf}. DWARF is a format mainly
used to describe debug information for executable files, essentially by
describing the source code of a program and mappings into to the compiled
binary. It is used in many modern compilers\footnote{Usages include GCC, Clang
  and Go} and is commonly used in combination with ELF binaries. It provides a
method to describe everything from source code locations, scopes, exception
handling, compilation units, linking information, etc. The primary building
block of DWARF information is the Debug Information Entry (DIE). It consists of
a DWARF tag that describes the kind of entry, and a set of attributes that
further defines the characteristics of an entry. Tags and attributes can be
nested which enables types to be as intricate as necessary. It is worth to
mention that DWARF is simply a \emph{format}, and while there are libraries to
help read and write the format, they are not part of the DWARF specification.

We do not use the full set of features provided by DWARF, rather we only use
DIEs for type information in ELF files. We would most likely also use DWARF to
facilitate proper debugging, but that is beyond the scope of this thesis.

DWARF DIEs are tree structures and we will use indention based notation to
present them. Listing~\ref{lst:design:dwarf:basic} shows how we can use DWARF to
express a 32-bit signed integer. Note that the name ``int'' has no special
meaning in DWARF, but consumers of DWARF may put special meaning into names.

\begin{lstlisting}[
  caption={A simple DWARF DIE describing a 32-bit signed integer},
  label={lst:design:dwarf:basic}]
<1> DW_TAG_base_type
        DW_AT_name = int
        DW_AT_byte_size = 4
        DW_AT_encoding = signed
\end{lstlisting}

In the above, there is a single tag with three attributes that describe the
type's name, size and encoding. The \code{DW\_TAG\_base\_type} essentially
designates a primitive type, one that is not a compound of one or more types,
and will usually be a leaf in the DWARF tree. For types that \emph{are}
compound, the approach is to store DIEs in an indexed table and reference types
by their index location in the table. In the above, the type's index is
designated by the \code{<1>}. By doing this no type ever has to be duplicated,
because any other types simply reference the same index. Tags start with the
\code{DW\_TAG\_} prefix while attributes use \code{DW\_AT\_}, all of which are
simply names for integer constants defined by the DWARF specification.

Listing~\ref{lst:design:dwarf:complex} shows a more complex example with a
subprogram (i.e.~a subroutine) that returns a pointer to the integer type
defined above.

\begin{lstlisting}[
  caption={A more complex example of DWARF types referencing each other},
  label={lst:design:dwarf:complex}]
<2> DW_TAG_pointer_type
        DW_AT_byte_size = 4
        DW_AT_type = <1>

<3> DW_TAG_subprogram
        DW_AT_name = foo
        DW_AT_type = <2>
        DW_AT_low_pc = 0x1
        DW_AT_high_pc = 0x10
\end{lstlisting}

It is shown how type \code{<2>} is defined as a pointer to the type \code{<1>},
which in C lingo is \code{int*}. Type \code{<3>} defines a subroutine whose name
is ``foo'' and returns a \code{<2>}, again in C lingo that would be \code{int
  *foo()}. This is the fundamental way that the DWARF format expresses types; it
is simple but very flexible and powerful.

There are scores of attributes and tags defined in the specification, but we do
not nearly use all of them. Because the format is so simple it is easy to add
new tags and attributes, simply by defining a new tag or attribute and a
corresponding numeric constant and documenting it. \thename{} defines some tags
TODO.

DWARF DIEs are, as mentioned, tree structures which is effective for expressing
types but is not very compact for storage,and can result in potentially unwieldy
data. There are more than one way to store the tree structure in a compact
manner, one of which we will describe here. TODO (which one does bfd use?).

\subsubsection{\thename{}'s Types System}

\thename{} provides a set of built-in types, encompassing both primitive types
as well as references, arrays and composite types. Further there is the concept
of a meta kinds which essentially wraps type definitions, type references,
subroutines, etc.~into a unified meta-data kind which are all stored in a single
table in the executable.

Compilers can map their primitive types to the ones provided by \thename{}, and
we expect that combinations of the available compound types to be sufficient for
expressing any other conceivable type however complex.

Below is a description of the various kinds of simple types provided.

\begin{description}
\item[Integer values] \hfill\\
  Includes all combinations of 8-bit, 16-bit and 32-bit, signed or unsigned
  integer numbers. E.g. {\tt Int32} and {\tt UInt16}.

\item[Boolean] \hfill\\
  A {\tt Boolean} type with the values `true` or `false`.

\item[Floating point values] \hfill\\
  32-bit or 64-bit floating point numbers, in the IEEE 753-1985 or IEEE 755-1985
  standard representation, respectively named {\tt Ieee32} and {\tt Ieee64}.

\item[Address] \hfill\\
  Unsigned integer value representing a memory address, named {\tt
    Address}. This is considered unsafe, and will therefore only be possible in
  code declared as `unsafe`.

\item[Abstract] \hfill\\
  Type which can take any value at run-time, named {\tt AnyType}. Large values
  will be boxed on the heap to make space for it.
\end{description}

% reference

The machine has a set of reference types, with the most fundamental being the
{\tt Reference <t>} type. This allows heap objects to be referenced from the
stack. As the name suggests, it will take a type of the object it referenced as
argument. In the opposite scenario, there is a {\tt StackReference <t>} type,
allowing passing references directly to stack references. This type may not be
boxed, nor passed to threaded tasks. This will save programs from major pitfalls
in concurrent programs (?).

To support variable argument calling conventions, letting a subroutine take an
arbitrary number of arguments, a special reference type is needed. This is done
through an {\tt ArgumentReference}, allowing a subroutine to iterate over its
given arguments. Also, to referencing a subroutine will be done through a {\tt
  CodeReference <signature>}, taking the signature which it implements. This is
a meta type, described below.

% arrays

The machine supports two types of arrays; a static and a dynamic type. Both will
be generic, taking the type of its values. This could be to other arrays,
allowing an array to be multi-dimensional. The static variant will also take a
lower and higher limit of values, aptly named {\tt StaticArray
  <t><lowerLimit><higherLimit>}. The dynamic variant will not take the limits,
as therefore named {\tt DynamicArray <t>}. To allow it to grow dynamically, it
will be instanced and put on the heap. It will therefore be referenced through a
{\tt Reference} type.

% composite

To support custom types, similar to a struct in the C programming language,
there will be a {\tt Composite <t+>} type. This will take an arbitrary number of
type arguments, describing its member. Though not visible from the type
signature, each member will have a name, allowing for convenient referencing (TODO).

Lastly, the machine will have a set of meta types, describing types by them
selves. These include {\tt Type}, {\tt Signature} and {\tt TypeSignature}. (TODO)


\subsection{Instruction Set Architecture}

The interface by which compilers and programmers interact with \thename{} is
defined by the Instruction Set Architecture (ISA). It is the language that the
machine understands and provides the means to utilize all of the above described
functionality. It is exactly the same as the x86 instruction set is to the x86
CPU architecture.

The \thename{} ISA is a set of instructions which are relatively low-level, as
compared to ordinary hardware implemented instruction sets. The abstraction
level of the \thename{} is aimed at being high enough for uncomplicated
implementation of high-level constructs but low enough for efficient code
generation. It can be divided into four overarching categories of functionality,
namely stack manipulation, numeric operations, control flow and heap related
instructions.

Most instructions require one or more parameters that define the exact behavior
of execution. By default, the arguments are read from the byte code stream, but
several instructions have \emph{indirect} variants that read one or more of
their arguments from elements on the stack instead of from the byte code
stream. This enables much more flexible and dynamic programs because arguments
to an instructions can be computed at run-time rather than having to statically
determine arguments.

There is a set of instruction \emph{prefixes} that modify how an instruction is
interpreted in terms of arguments, safety and exception handling. Technically
they are simply predefined byte values that are parsed before an instruction is
executed. For instance, integer arguments are by default read as 8-bit values,
but may be extended to 32-bits by providing the \code{large} prefix. Another
prefix, \code{noOverflow}, will make overflowing arithmetic throw an
exception. Prefixes generally only take effect for a single instruction and are
reset at the end of an execution cycle.

The following describes each of the four categories of instructions in detail
and highlights some of the most interesting examples.

\subsection{Stack Manipulation}

The stack manipulating instructions facilitate the functionality described in
Section~\ref{sec:design:stack-mgmt}. It can be argued that all instructions are
stack manipulations since almost every instruction will leave the stack
changed. However we define stack manipulations as the instructions that either
add or remove an element on the stack without causing \textit{side effects}. A
side effect is essentially anything that results in changes in the heap. Control
flow and numeric instructions are not considered stack manipulations either.

The stack manipulating instructions mostly consist of set of push-instructions
that push values onto the stack. What that values represents and how it is
obtained differs between the individual instructions, the most fundamental of
which are described below.

\begin{description}

\item[\code{pushElement}:]

  Fetches an element from upward into the stack and pushes a copy back onto the
  stack. This is useful, and indeed required, for loading arguments in
  sub-routines and can be used by a compiler to implement local variables. We
  further provide the opposite mechanism; an instruction for storing a value
  back into an element upward into the stack, i.e.~replacing the value of a
  stack element. The source element and the target element must have either the
  same or unifiable types.

\item[\code{pushLiteral}:]

  Takes a type argument in the form of an index into the type table defined in
  the executable. The next byte (or bytes depending on the presence of the
  \code{large} prefix), in the byte code stream is parsed as a literal value
  which is pushed onto the stack along with the type.

\item[\code{pushConstant}:]

  Works just as \code{pushLiteral} except that the value is read via an index
  into the constant table rather than read from the byte code stream. This
  allows arbitrarily large literal values to be pushed to the stack, required
  for any value bigger than 32-bits. This is both useful for 64-bit numeric
  values but can also be used for static arrays and object literals.

\item[\code{pushMeta}:]

  Meta data is, as mentioned in Section~\ref{NEEDED}, a wrapping kind
  encompassing type information, field information and sub-routine
  information. The \code{pushMeta} takes an index into the meta data table and
  places the corresponding value on the stack. This is required for many of the
  indirect instruction variants, eg.~where a type reference is read from the
  stack.

\item[\code{pushUninitialized}:]

  Creates a new stack element with a provided type and reserves space on the
  stack according to the type. The value of the element is initialized to 0
  regardless of the type, meaning that it is not guaranteed to be a valid value
  before properly initialized. A use case is for allocating local variables and
  out parameters (for return values) for use in sub-routines.

\item[\code{pop}:]

  The operation that removes one or more values from the stack. It takes an
  argument which defines the number of elements to remove.

\end{description}

%pop

\subsubsection{Numeric and Logical Operations}

\thename{} supports the usual range of logic operations including \emph{and},
\emph{or}, \emph{xor} and bit-wise shifts along with the arithmetic operations
of addition, subtraction, multiplication, division and remainder. Unlike JVM we
only provide a single instruction per operator which takes care of proper type
detection and potential conversion. That makes it much simpler to write generic
arithmetic code without having to create specialized versions for each
combination of types.

The \code{shl} and \code{shr}, for shift left and shift right, performs logical
shifts. That is, they displace shift bits in the given direction and simply pads
with 0-bits in the opposite direction. The \code{ashr} instruction performs an
\textit{arithmetic shift} to the right which means that the sign-bit of an
integer value is extended. The sign-bit is simply the left-most bit (we are
using big endian representation) and is extended regardless of the sign of the
type. In effect that gives the following results (number are in binary
representation):
\[
0b1001 shl 2  = 0b0100 \\
0b1001 shr 2  = 0b0010 \\
0b1001 ashr 2 = 0b1110 \\
\]

It is worth noting that we do not provide an arithmetic left shift instruction
because it is mostly redundant since there is right-most bit is insignificant
and does not usually make sense to extend.

Numeric values can be converted other numeric types using the \code{convert}
instruction. It takes a type as a parameter and will convert a value on the
stack to the provided one. The \code{convert} instruction is only defined for
numeric values and is thus not meant or able to be used for converting and
casting any other type.

In addition all arithmetic instruction have so-called saturated variants which
compute values that are saturated within the bounds of a given type in case of
overflowing or underflowing results. Consider adding the values $255$ and $2$
both as unsigned 8-bit integral values (UInt8). The maximum value of a UInt8 is
$255$ meaning that $255+2=257$ will result in an overflow. Using the normal
\code{add} instruction the result will be truncated to $1$ because the excess
bytes are ignored, but with the saturated variant the result will remain $255$
because the overflow is detected and the value is capped at the maximum
possible. The same is true for underflowing values with both signed and unsigned
values.

\subsubsection{Heap Related Instructions}

The heap related instructions form the interface to the object model of
\thename{}. Several of the instructions are closely related to the composite
type (see Section~\ref{NEEDED}) due to the fact that a composite type is used to
describe and inspect a heap object's layout in memory.

To create a new heap object, space must be allocated on the heap. This is done
with the \instr{new} instruction which takes a type representation and delegates
the task of actual memory allocation to the garbage collector. A reference to
the newly created memory region is then pushed to the stack. Constructor code
can be run to initialize the object and finally the \instr{release} instruction
must be executed so as to mark the object ready for garbage collection. Note
that releasing an object does \emph{not} free its memory region, but rather
``releases'' the object into the hands of the garbage collector. This pattern is
similar to how Objective-C manages object creation, namely by use of the
\code{alloc} and \code{init} methods, which allocates and initializes an object
it, respectively.

There is a whole family of instructions for getting and setting fields of
objects, which correlates to the members of a composite type. The general
structure of field instructions is the same, the difference lies in \emph{where}
the field is retrieved from. Fields can be retrieved from any location that
defines a composite type, which means either a stack element, heap object or the
global scope. Following is the description of the simplest forms, namely the
global scope, but the information applies to the whole set of field
instructions:

\begin{description}

\item[\code{pushField}:]

  Takes a field reference as argument and looks up the corresponding memory
  location in the global scope. The value of the field is pushed to the stack as
  a stack element with the same type as the field's type.

\item[\code{popField}:]

  Takes a field reference and a value of the \emph{same type}. The field
  reference is used to look up a memory address into which the value is copied.

\end{description}

Finally there is the \instr{box} and \instr{unbox} instructions that are used to
move values between the stack and the heap. \instr{box} takes value and moves it
to the heap by wrapping it in a heap object. The type of the heap object will be
the same as the value's type but wrapped in a reference type. Correspondingly,
\instr{unbox} takes a reference to a heap object, unpacks it's value and pushes
it onto the stack with the type that the consumed reference was referencing.

\subsubsection{Control Flow}

\thename{} supports different forms of control flow mechanisms. First there are
sub-routines which are invoked via the \code{invoke} instruction. It takes an
index into the meta table which is expected to be a sub-routine definition
containing information about the sub-routine's byte address bounds and parameter
types. When executed, the current thread's program counter will be set to the
lowest address of the sub-routines byte address bounds. Execution will then
continue from there until a \code{return} instruction occurs which will set the
program counter back to the instruction following the \code{invoke} that
initiated the call (i.e. the address of the callee). The same call/return
mechanism can be used without a defined sub-routine by use of \code{call}, which
takes a byte address offset from where execution will continue.

For performance critical scenarios we provide a switch/case instruction that
does constant time case branching by using a jump table. The concept is
relatively simple and is an example of a variable-length instruction. The
\code{switch} instruction has the following format:

\code{switch <UInt8>target\_count, <Int8>target\_0, \ldots, <Int8> target\_N}

The \code{target\_count} defines the number of entries in the jump table and is
followed by exactly that amount of target byte addresses. When executed, the top
of the stack is popped and its value is used as index for a lookup into the list
of targets, which is an O(1) operation. The last target, i.e. \code{target\_n},
is treated as default case, meaning that if the value on the stack lies without
the range $[0;n]$ it will be chosen instead of looking up a value. This makes a
jump to a case branch of a switch statement very efficient, but has certain
limitations. If case values are not adjacent, ``fake'' branch targets must be
inserted, usually pointing to the default target. The instruction's length is
equal to the maximum case value, meaning that if there are just two cases, 1 and
1000, then there will be 998 redundant bytes in the instruction, which is
obviously unacceptable. In that case it will be a better choice to use a series
of conditional branches.



\subsection{Usage Examples}

% local vars
% sub-routines

% Examples of how to map things to our model ()
%% Classes
%% Inheritance
%% HoF
%% Polymorphism

% Type system
%% Representation (DWARF)
%% Typed instructions
%% Run-time type checking
%% Dynamic types
%%% !!!

% Executable format
%% ELF
%%% It's tried-and-true
%%% Everybody else uses it so we can port stuff easier

% Object model
%% Hash maps
%%% The efficiency of them
%% Dynamic dispatches

% Threading
%% Tasks and transactions
%%% Shared stack elements
%% ?!
%% Thread Pool
%%% Resource thrashin

% Closures

% Memory management
%% Garbage collection

% ISA
%% Highlight the most important instructions, explain how they facilitate the above

% Mem model
%% Standard heap and stack, stack per thread, global heap
%%

%%% Local Variables:
%%% mode: latex
%%% TeX-master: "../report"
%%% End:
