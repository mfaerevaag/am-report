The problem that \thename{} attempts to solve is a complex matter that spans a
wide range of areas in the field of computer science. As a result, our work on
the implementation has involved a great deal of different types of work, varying
from instruction set design and executable file formats, to efficient hash maps
and C code architecture.

Along the way we have gained a thorough knowledge of the internals of abstract
machines and code interpretation in general. We have come to realize that it is
no simple task to build a machine that is capable of expressing most of the
modern programming language paradigms in a unified manner. In addition we have
discovered the immense value and assurance that extensive tests provides.

Implementing a machine like \thename{} has been an interesting experience in
terms of working on a non-trivial and relatively low-level software project
using an agile workflow that allowed us to cope with regular changes.

We have achieved a result that can serve as evidence that the vision of
\thename{} is indeed possible. Our benchmarkings revealed that the run-time
performance, in most cases, is sub-optimal for practical uses, but does not fall
far behind the Java Virtual Machine in isolated cases. The general maturity of
the machine is not on par with the existing industrial strength systems, but it
does do a good job of implementing the fundamental semantics in a different way
that could be the cornerstone for future work.

%%% Local Variables:
%%% mode: latex
%%% TeX-master: "../report"
%%% End:
