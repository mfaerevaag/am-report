\subsection{Tools and Languages}

We have implemented \thename{} in the C programming language, more specifically
using the C99 standard with GNU extensions by use of GCC\footnote{The GNU
  Compiler Collection: \url{https://gcc.gnu.org}}. There are both pros and cons
of using a relatively low-level language (as compared to other available
languages), but we found that the benefits outweighed the disadvantages.

First of all C is extremely portable; GCC is available on \textit{a lot} of
architectures, and all modern operating systems ship with a C compiler which
essentially is all that is required to build \thename{} for a new platform. The
standard libraries and architecture specific functions available differ with
each machine and device, so we aim at relying on as few external libraries as
possible and provide wrappers for the ones we use so that the implementation can
be ported quickly.

Secondly we do not want to rely on a garbage collector provided by languages
like C++ and Java. We need to full control over where each bit of data is
stored, mostly for performance reasons both in terms of memory footprint and
execution speed.

Third, speed is of the essence and C is notoriously fast, given that the code is
well-written.

There are however downsides to using C. Because we are not veterans of C and due
to its low-level nature it is inherently more difficult to develop features that
could be implemented in a few lines of code in a higher-level language, and
memory has to be tracked meticulously to prevent memory leaks. We have found
that it is a great exercise for becoming proficient with programming in general
because it provides us with knowledge of the mechanisms that lie behind concepts
that are generally taken for granted.

For finding such memory leaks we use Valgrind~\footnote{Valgrind:
  \url{http://valgrind.org/}}, and for profiling and benchmarking we use GNU
gprof~\footnote{GNU gprof: \url{https://sourceware.org/binutils/docs/gprof/}}.

For debugging we use gdb via GNU Emacs~\footnote{GNU Emacs:
  \url{http://www.gnu.org/software/emacs}} and for analyzing memory usage and
fixing memory leaks we use the Valgrind instrumentation tool\footnote{Valgrind:
  \url{http://valgrind.org}}.

The code is extensively documented using Doxygen~\footnote{Doxygen:
  \url{http://www.stack.nl/~dimitri/doxygen}} style comments.

\textbf{Note:} The code snippets presented in the following sections does not
always correspond directly with the code found in the actual
implementation. They carry the gist of what we want to communicate but generally
avoids unnecessary and redundant information not relevant to the individual
cases.

\subsection{Code Style}

The C languages frees the programmer to structure code and data in about as many
ways as one can imagine. This can be both a blessing and a curse, and warrants
some agreed upon conventions to maximize consistency.

For a module of functionality we use a general pattern that consists of a struct
containing the state or data of a piece of machinery, and corresponding prefixed
functions that take such a structure as argument and operate on its
data. Stacks, for instance, are implemented with a \code{stack} struct, and the
functions \code{stack\_init}, \code{stack\_pop}, \code{stack\_free}, etc. This
essentially mimics object orientation from other languages (C++, Java), but has
the benefit of most functions being \term{pure}, i.e.~a function that only
depends on its input, making testing and reasoning about the code easier.

We use a few naming conventions that will be present in the code as well as code
snippets found in this thesis: %TODO might be useless

\begin{itemize}
\item We use underscore to separate multi-word variable and function names.
\item Structs are defined with \code{\_s} suffix and typedef'd to a name
  without.\\Eg: \code{struct stack\_s \{...\}} and \code{typedef struct stack\_s
    stack}.
\item Enums are defined with \code{\_e} suffix and typedef'd to a name
  without.\\Eg: \code{struct opcode\_e \{...\}} and \code{typedef enum opcode\_e
    opcode}.
\end{itemize}

\subsection{Core Infrastructure}
\label{sec:implementation:core}

% Very general architecture (theres a vm, a file reader, instruction codes, etc)
% What's in the VM state and Thread state?
% How code bytes are handled
% Instruction cycles
% Internal error handling
% Logging/debugging facilities
% Graceful exits / clean up
% Command line arguments

% thread
% instructions
% stack
% types
% gc interface
% symtab, strtab, ttable

The infrastructure of \thename{} is the code that lays the basis for interaction
between the individual parts of the system. It is a collection of machinery
which hand control back and forth between each other, such as the instruction
cycle, thread management and exception handling. Besides the machine's exposed
exception handling system there is an internal error handling mechanism which
allows errors within the machine to be handled as gracefully possible.

The machine is divided into several modules of which \code{am}, for abstract
machine and implemented in \code{am.h/c}, is the most overarching. It defines a
structure, the \code{am\_state} struct, which contains data that is generally
available in a machine-wide context. We say ``generally'' because some parts of
the system are by design ignorant of their context. A string table, for
instance, is not aware of which thread is using it, and instructions do not know
of the abstract machine state. The most important members of \code{am\_state}
are a pointer to the code bytes and a pointer to the main thread. The state is
initialized by loading in the code %TODO

During the initialization phase of the machine approximately the following steps
are carried out, in order: %TODO

\begin{enumerate}
\item Parse command-line arguments
\item Set up internal error handling mechanisms
\item Initialize global abstract machine state
  \begin{enumerate}
  \item Load byte code from the executable
  \item Load and parse information tables
  \end{enumerate}
\item Initialize root scope
  \begin{enumerate}
  \item Load the the run-time library functions
  \end{enumerate}
\item Initialize main thread
  \begin{enumerate}
  \item Allocate stack
  \end{enumerate}
\item Begin the instruction cycle in the main thread
\end{enumerate}

We will briefly explain each step in the following sections.

\subsubsection{Command-line Arguments}
\thename{} will eventually be run through an executable binary, most likely in a
command-line environment. The most common, and arguably the most efficient, way
of customizing parameters and execution options is through command-line
arguments, given directly to the executable when run. These arguments can be
extended to do a number of things, but essentially we only require the name of
the binary file to be executed and optionally customizing the level of
verbosity, i.e. the amount of logging information to display.

We follow common conventions by having two special arguments for displaying
useful meta information about the \thename{} binary. These include the {\tt
  --help} argument for printing information on the different arguments, and {\tt
  --version}, for simply printing the version of the binary.

For ease of implementation we use the Getopt library~\footnote{Getopt (libc):
  \url{http://www.gnu.org/software/libc/manual/html_node/Getopt.html}}, included
in C's Standard Library.

\subsubsection{File Reading}
The first thing the machine does upon start up is to load various parts of the
executable file into memory. %TODO

\subsubsection{Instruction Cycle}
% general loop (check with whats in design)
% instruction functions signature
% role of instructions vs vm (who fetches what)
% byte code address is into elf, ie offset from first byte when loaded

The fundamental means of operation in the machine is the execution of
instructions that are parsed and interpreted from byte code. Each thread runs
their own cycle, the essence of which is shown in
Listing~\ref{lst:implementation:instruction-cycle}

\begin{lstlisting}[%
  caption={Pseudo representation of the instruction cycle},%
  label={lst:implementation:instruction-cycle}]
while (thread is running) {
  fetch opcode
  fetch potential arguments
  execute instruction corresponding to opcode
  update program counter
}
\end{lstlisting}

Each task is performed by different modules of the machine. The abstract
machine, \code{am}, module is responsible for reading byte code including
reading the opcodes, parsing arguments and in turn executing the function that
implements the individual functions. The instruction module,
\code{instruction.h/c} is where instruction routines are defined, one function
per instruction. All instruction functions take a \code{thread\_state*} argument
that they use to manipulate the stack, change the program counter, access scopes
and so on. Generally the abstact machine does not do look ups into information
tables and other logic, rather its responsibility is to read and parse the byte
code and delegate the rest of the work to other modules.

As an example consider the following byte code stream:

%TODO needs correct opcode value
\code{0: 0xFC, 1: 0x10, 2: 0x00, 3: 0x00, 4: 0x05, 5: 0x39, ...}

And let the program counter be $pc = 0$. The machine will carry out the following:

\begin{enumerate}
\item Read instruction prefixes
  \label{item:read-prefix}
  \begin{enumerate}
  \item Look for bytes that are valid prefix values which is any value greater
    than or equal to \code{0xFC} (decimal 252).
  \item The byte at $pc=0$, \code{0xFC}, is a valid prefix, namely the
    \code{large} prefix.
  \item Set the \code{pre\_large} flag in the executing thread's state and
    increment program count, $pc \leftarrow pc + 1 \Rightarrow pc= 1$
  \item The byte at $pc=1$, \code{0x10}, is not a valid prefix, thus stop
    reading prefixes.
  \end{enumerate}

\item Read the opcode which is always a single byte. The byte at $pc=1$ is
  \code{0x10} which corresponds to the \instr{pushConstant} instruction.

\item Increment program counter, $pc \leftarrow pc + 1 \Rightarrow pc = 2$

\item Fetch arguments
  \begin{enumerate}
  \item \instr{pushConstant} takes an argument, an integer value representing a
    constant table index.
  \item The \code{pre\_large} flag is true, so we must read the argument as a
    32-bit (four bytes) integer.
  \item Interpret the next four bytes as a big endian integral value. The four
    bytes are \code{0x00, 0x00, 0x05, 0x39} which is interpreted as the value
    \code{0x539} (decimal 1337).
  \item Increment program counter by four bytes,
    $pc \leftarrow pc + 4 \Rightarrow pc = 6$
  \end{enumerate}

\item Call the instruction function for \instr{pushConstant} with the argument
  $1337$.

\item Go to~\ref{item:read-prefix}.

\end{enumerate}

The implementation of concrete instructions is detailed in
Section~\ref{sec:implementation:instr}.

\subsubsection{Debugging}
Logging macros
Debug functions (print\_stack)


\subsection{Stacks}
\label{sec:implementation:stacks}
A stack is in itself a fairly simple construction that allows values to be
pushed to an popped from an allocated memory region. Stack elements are stored
and retrieved by the Last-In, First-Out (LIFO) principle.

\subsubsection{Stack elements}

Each element on a stack holds information about both the \textit{declared} and
\textit{actual} type of data that it contains. Details on this is described in
\ref{NEEDED}. The data part of a stack element is a simple array of bytes,
stored in big endian order, i.e.~the most significant bit is at the lowest
memory address.

\subsubsection{Memory Layout}

Arguably the most intuitive way to construct a stack is to use a simple array of
a fixed initial size and handle operations by keeping track of the number of
elements in the stack. A push operation would thus increments the number of
elements and assigns a value to the new top-most slot. A pop operation will
correspondingly decrement the number of elements in the stack and potentially
return the element that is being removed. When the number of elements in the
stack exceeds the initial size of the containing array more memory must be
allocated. This can be done simply by use of \code{realloc} where the common
approach is to double the amount of allocated memory, to adapt to quickly
growing stacks. However, \code{realloc} can be a performance issue because it
will create a new allocation, copy over the current data and free the old
allocation, in the case the there is no more space directly following what is
being reallocated~\cite{man-realloc}. This is arguably an expensive operation,
and since the stack is by far the most frequently used means of operation in
\thename{}, it is critical that it performs well.
% TODO: agree ^ realloc not _always_ moves memory, but etc?

An important thing to consider is that the content of a stack element depends on
the type of the element, and thus have varying sizes. An element whose type is
\code{Int8} requires a single byte for storage while an element of type
\code{Int32} requires 4 bytes of storage and static arrays can be arbitrarily
large. Since we do not want to waste memory on redundant bytes, that prevents us
from implementing stack elements as fixed size structures laid out in an array
as described above. We have solved the problem by implementing stack elements as
a singly linked list going from the top element down to the bottom. The memory
used by the stack should be as contiguous as possible, to avoid a large amount
of allocations and deallocations, and the elements in the linked list should
reside one after another in memory. The actual content of an element is stored
\textit{in between} the linked list nodes, and the \code{next} pointer simply
points to the memory address following its content data
bytes. Figure~\ref{fig:implementation:stack-layout} shows how the stack elements
are laid out, with arrows denoting pointers.

\begin{figure}[H]
  \centering
  \begin{drawstack}[every initial by arrow/.style={*->}]
  \startframe{}
  \cell{Declared type} \cellcom{\texttt{rtti*}} \coordinate (next) at (currentcell.east);
  \cell{Actual type} \cellcom{\texttt{rtti*}}
  \cell{Next} \cellcom{\texttt{stack\_element*}}
  \cell{Content size} \cellcom{\texttt{int}}
  \padding{2}{Content}  \cellcom{\texttt{char[]}}
  \finishframe{\texttt{stack\_element}}

  \startframe{}
  \cell{Declared type}  \cellcom{\texttt{rtti*}}
  \cell{Actual type}  \cellcom{\texttt{rtti*}}
  \cell{Next} \cellcom{\texttt{stack\_element*}} \coordinate (el2) at (currentcell.east);
  \cell{Content size} \cellcom{\texttt{int}}
  \padding{2}{Content}  \cellcom{\texttt{char[]}}
  \finishframe{\texttt{stack\_element}}
  \startframe{}

  \draw[->, thick, bend right=20] (el2) edge (next);
\end{drawstack}

% \begin{drawstack}
%   \startframe
%   %   \cellcom writes something on the right-hand side of a cell.
%   \cell{loc2} \cellcom{-8(\%ebp)}
%   \cell{loc1} \cellcom{-4(\%ebp)}
%   %   \esp and \ebp are stack pointer and base pointer in Pentium.
%   %   These macros are simple shortcuts for \cellptr{...}
%   \cell{Sauvegarde \%ebp} \esp \ebp
%   \cell{@ retour} \cellcom{4(\%ebp)}
%   \finishframe{fonction\\ {\tt f}}
%   \startframe
%   \cell{} \cellcom{8(\%ebp)}
%   \cell{}
%   \finishframe{fonction\\ {\tt main}}
% \end{drawstack}

% \section{Padding}

% \begin{drawstack}
%   \cell{above padding}
%   \padding{3}{nothing here}
%   \cell{below padding}
% \end{drawstack}

% \section{Below/Above stack pointer}

% \begin{drawstack}
%   \cell{Top}
%   \cell{Below top}
%   %   \bcell is just like \cell, but in a different color.
%   \bcell{Above bottom} \cellptr{Stack pointer here}
%   \bcell{Bottom}
% \end{drawstack}

% \section{Highlighting some cell}

% \begin{drawstack}
%   \cell{Uninteresting cell}
%   \cell{Interesting cell} \cellround{Yes, this one!}
% \end{drawstack}

% \section{Structures without a stack structure}

% \begin{tikzpicture}
%   \draw (3, -1) node (Otm) {
%   \begin{tabular}{c}
      %       Object\\vtable
      %     \end{tabular}
      %       };

      %       \drawstruct{(0,0)}
      %       \structcell[freecell]{~} \coordinate (Atm) at (currentcell.east);
      %       \structcell[freecell]{\texttt{@Object.equals()}}
      %       \structcell[freecell]{\texttt{@code A.m()}}
      %       \structcell[freecell]{\texttt{@code A.p()}} \coordinate (A) at (currentcell.west);
      %       \structname{
      %       \begin{tabular}{c}
      %       A's vtable
      %     \end{tabular}
      %       }

      %       \drawstruct{(-4,-3)}
      %       \structcell[freecell]{} \coordinate (Btm) at (currentcell.east);
      %       \structcell[freecell]{\texttt{@Object.equals()}}
      %       \structcell[freecell]{\texttt{@code A.m()}}
      %       \structcell[freecell]{\texttt{@code B.p()}}
      %       \structcell[freecell]{\texttt{@code B.q()}}
      %       \structname{B's vtable}

      %       \draw[->] (Btm) -- (A);
      %       \draw[->] (Atm) -- (Otm);
      %       \end{tikzpicture}

      %       \section{Structures and stack together}

      %       \begin{tikzpicture}[scale=.8]

      %       \stacktop{}
      %       \separator
      %       \cell{\texttt{p3}}        \cellcomL{11(GB)} \coordinate (p3) at (currentcell.east);
      %       \separator
      %       \cell{\texttt{p2}}        \cellcomL{10(GB)} \coordinate (p2) at (currentcell.east);
      %       \separator
      %       \cell{\texttt{p1}}        \cellcomL{ 9(GB)} \coordinate (p1) at (currentcell.east);
      %       \separator
      %       \cell{\texttt{@P3D.diag}} \cellcomL{ 8(GB)}
      %       \cell{\texttt{\footnotesize @Object.equals}} \cellcomL{ 7(GB)}
      %       \cell{\texttt{3(GB)}}     \cellcomL{ 6(GB)} \coordinate (T1) at (currentcell.east);
      %       \separator
      %       \cell{\texttt{@P2D.diag}} \cellcomL{ 5(GB)}
      %       \cell{\texttt{\footnotesize @Object.equals}} \cellcomL{ 4(GB)}
      %       \cell{\texttt{1(GB)}}     \cellcomL{ 3(GB)} \coordinate (T2) at (currentcell.east);
      %       \separator
      %       \cell{\texttt{\footnotesize @Object.equals}} \cellcomL{ 2(GB)}
      %       \cell{\texttt{null}}      \cellcomL{ 1(GB)}
      %       \cell[draw=none]{Stack}


      %       \drawstruct{(5,1)})
      %       \structcell{z=2,5}
      %       \structcell{y=2,5}
      %       \structcell{x=2,5}
      %       \structcell{.} \coordinate (O1) at (currentcell.west);
      %       \coordinate (O1l) at (currentcell.south);

      %       \drawstruct{(9,-3)}
      %       \structcell{y=1}
      %       \structcell{x=1}
      %       \structcell{.} \coordinate (O2) at (currentcell.west);
      %       \coordinate (O2l) at (currentcell.south);

      %       \draw[->] (p3) -- (O1);
      %       \draw[->] (p2) -- (O1);
      %       \draw[->] (p1) -- (O2);

      %       \draw[->] (O1l) .. controls (O1 |- T1) .. (T1);
      %       \draw[->] (O2l) .. controls (O2 |- T2) .. (T2);

      %       \draw (10,-10) node{Heap};

      %       \end{tikzpicture}



      %       \section{Using tikzpicture instead of drawstack}

      % %       The environment drawstack is basically a syntactic sugar for
      % %
      % %       \begin{tikzpicture}[#1]
      % %         \stacktop{}
      % %         ...
      % %         \stackbottom
      % %       \end{tikzpicture}
      % %
      % %       You can use the above syntax for more flexibility.

      %       \begin{tikzpicture}[scale=0.8]
      %       \small
      %       \stacktop{}
      %       \cell{My cell}
      %       \stackbottom{}
      %       \end{tikzpicture}

      %       \section{Changing style}

      %       {% tikzstyle will be local to this {...}
      %       \tikzstyle{freecell}=[fill=blue!10,draw=blue!30!black]
      %       \tikzstyle{occupiedcell}=[fill=blue!10!orange!10,draw=blue!30!black]
      %       \tikzstyle{padding}=[fill=yellow!20,draw=blue!30!black]
      %       \tikzstyle{highlight}=[draw=orange!50!black,text=orange!50!black]

      %       \begin{drawstack}
      %       \cell{Uninteresting cell}
      %       \cell{Interesting cell} \cellround{Yes, this one!}
      %       \bcell{bcell}
      %       \padding{2}{Padding}
      %       \end{drawstack}
      %       }

      %       \section{Example: Computing Factorial}

      %       \begin{drawstack}[scale=0.8]
      %       \startframe
      %       \cell{N = 1}
      %       \cell{...}
      %       \finishframe{fact(1)}
      %       \startframe
      %       \cell{N = 2}
      %       \cell{...}
      %       \finishframe{fact(2)}
      %       \cell{$\vdots$}
      %       \startframe
      %       \cell{N = 5}
      %       \cell{...}
      %       \finishframe{fact(5)}
      %       \end{drawstack}

  \caption{Stack memory layout}
  \label{fig:implementation:stack-layout}
\end{figure}

The C language does not support variable length arrays, which means that an
array must have its length declared, otherwise the type is regarded as
incomplete. However, line~\ref{code:stack-element:flexible-data} in
Listing~\ref{lst:implementation:stack:element} shows a feature known as a
\term{flexible array} that allows the last field of a struct to be declared as a
size-less array. We take advantage of exactly that so that we can easily
interpret the \code{data} field of an element as an array of bytes of any size.

What is left to solve is the problem of a stack growing beyond its initial
size. As mentioned, reallocation of the memory used by the stack works, but
results in poor performance. To mitigate the issue we have modelled the stack as
a linked list of allocated memory regions which we call \textit{segments}. Each
segment is essentially a stack in itself and is implemented exactly as described
above. The benefit is that when a segment becomes full, a new segment is simply
allocated and wired into the linked list, thus preventing the need for a
reallocation. The exposed interface of the stack consists of a set of stack
operation functions that essentially maps operations onto segments by selecting
the appropriate segment, and potentially what index of its elements that will be
the target of a given operation. Thus the segments are hidden in the
implementation and are solely internal data structures which users of the stack
are ignorant of.

\begin{remark}
  A segmented stack is not a novel idea. They are used by the Go programming
  language to facilitate large amounts of unbounded stacks (for concurrent
  sub-routines called goroutines). The Rust programming language did as well but
  later abandoned them. There are arguments for not using them because of the
  overhead generated by extra work needed when stack limits are
  exceeded\cite{rust:segmented-stack, go:segmented-stack}.
\end{remark}

Segments are implemented as a doubly linked list as shown in
listing~\ref{lst:implementation:stack:segment}. Each segment keeps track of its
length (i.e.~number of elements current in the segment) and the top element. The
\code{content} field is the region allocated for storage of elements. The
\code{data\_cursor} is maintained to always point to the byte following the top
element's content bytes and is useful when pushing a new element. The
\code{size} field simply represents the size of the memory region allocated for
the segment and is used to check whether the segment has room for more elements.

\begin{figure}[h]
  \centering
  \begin{lstlisting}[language={[ANSI]C},%
    caption={Structure defining the stack},%
    label={lst:implementation:stack:element}]
struct stack_element {
    /* points to next element in the upward direction */
    struct stack_element_s *next;

    int actual_type;
    int declared_type;

    byte data[]; (*@\label{code:stack-element:flexible-data}@*)
}
  \end{lstlisting}
\end{figure}

A segment has a limited amount of space for elements but the stack itself is
unbounded as a result of the segment list; there is no limit to how many
segments can be allocated, except the amount of physical or virtual memory
available to the machine.

Stack elements can be arbitrarily large. For instance fixed size arrays stored
on the stack are as large as their type indicates, and might be bigger than the
defined segment size. Thus, before pushing an element (as shown in
Listing~\ref{lst:implementation:stack:operations}) we check whether the element
can fit into a segment, and if not a segment is allocated with the size
explicitly given. The new segment is allocated the required size for the element
plus the default segment size as to allow for more elements to be pushed and
minimize the segment handling overhead.

\subsubsection{Caching Stack Segments}

A new segment is allocated when the last active segment becomes full, but is not
immediately freed when it becomes empty. Instead a number of segments are cached
to facilitate series of stack operations that operate right between the limit of
two segments. Consider a loop of code that repeatedly pushes four elements to
the stack only to pop them off moments after. If there are two slots left in the
last segment when the loop begins, a new segment must be allocated during the
course of one loop iteration. However when the iteration pops the values off
again the new segment will become empty, which without the cached segments would
result in the segment being freed. With the cached segments the newly allocated
segment is saved and is simply rewired into the stack when it is needed again.

When there are more inactive segments than defined by
\code{STACK\_NUM\_CACHED\_SEGMENTS} a segment will be freed. This is a trivial
operation because the all of the segment's data and elements' data lie within
the segment's memory allocation, so it is simply freed and there is no need to
free individual elements.

%%% Local Variables:
%%% mode: latex
%%% TeX-master: "../report"
%%% End:


\subsection{Execution Model}
% \label{sec:implementation:execmodel}
% %%% Local Variables:
%%% mode: latex
%%% TeX-master: "../report"
%%% End:

\subsubsection{Instruction Prefixes and Suffixes}

TODO: move elsewhere?

Instructions can be prefixed or suffixed to invoke special behavior. TODO.

The {\tt noOverflow} prefix will enabled the machine to throw exceptions if
overflow occurs when executing the following instruction. This might for
instance be desired when executing arithmetic instruction, to ensure that the
arithmetic operation does not induce overflow which could induce unwanted
behavior.

Overflow is detected differently for different arithmetic operations and operand
types. This is required as different as types have different maximum and minimum
values. We will briefly explain the challenge of handling overflow in different
scenarios and how \thename{} currently, and imperfectly, tackles them.

Looking at how to detect overflow when doing arithmetic operations on {\em
  signed integer} operands, we have to take the size of the the operands' type
into account. For instance, an {\tt Int8} can only represent integer numbers in
the rage of -128 to 127. Knowing the type's maximum and minimum value, we can
deduce whether the operation will induce over- or underflow on operand $a$ and
$b$. In the case where the operation is addition, we check if operand $b$ is
larger than zero {\em and} operand $a$ is lesser than the sum of the type's
maximum value and $b$'s inverse. This can be denoted in the following
mathematical formula:
\label{eq:overflow}
\begin{equation}
  overflow = (b > 0) \wedge (a > max - b)
\end{equation}

In a similar fashion we can detect underflow by:
\begin{equation}
  underflow = (b < 0) \wedge (a < min - b)
\end{equation}

If the operation is subtraction, we just change the plus operator to minus, in
the two equations. In the case of multiplication we use division, etc.

For unsigned integers the case is a little simpler, as the operand cannot be
negative. Therefore, when doing addition, underflow can per definition not
occur:
\begin{equation}
  a + b > 0, \qquad when\ a > 0, b > 0
\end{equation}

For substitution, we only needs to check if the $b$ operand is greater
than $a$, which will result in a negative number, thus inducing underflow.
\begin{equation}
  a - b < 0, \qquad when\ a > 0, b > a
\end{equation}

These fairly simple solutions has shown through tests, to work well where the
operand precision is not {\tt Int64} or {\tt UInt64}. The reason for this is
that all overflow check function promote each operand to an {\tt int64\_t} or
{\tt uint64\_t}, which prevents overflow of all operands with lower
precision. In the case of 64-bit integers, there is a risk of the check it self
to overflow. More precisely, $a < min - b$ will overflow if $b$ is
positive. There is not good solution for this, other than to use a higher
precision number for the tests. As C does not have any built-in data types of
larger precision, one would have to use multi-precision numbers which would
require some time to implement. We have chosen to note the short comings of the
current solution and focus our attention else where (TODO).

Both
GCC~\footnote{\url{https://gcc.gnu.org/onlinedocs/gcc/Integer-Overflow-Builtins.html}}
and
Clang~\footnote{http://clang.llvm.org/docs/LanguageExtensions.html\#checked-arithmetic-builtins}
has built-in functions for doing integer arithmetic while also detecting
overflow. GCC's implementation (TODO: maybe also Clang) does this by converting
the two operands into infinite precision numbers, where after doing the the
operation on the promoted operands. The result is then cast to the original
precision. If the casted result does not equal the infinite precision result,
overflow has occurred, which is indicated by the functions return value. We have
chosen to not use these to hold the implementation as library independent as
possible TODO.

Detecting overflow for floating-point precision numbers is somewhat more
challenging, due to the more complex nature of its implementation. The C
Standard Library has built-in constants for finding the a float's maximum value;
{\tt FLT\_MIN} and {\tt FLT\_MAX} for single precision, {\tt DBL\_MIN} and {\tt
  DBL\_MAX} for double precision. These could in theory be used in the same
fashion as above, creating an efficient solution, given that we could do
built-in arithmetic operations on them. This is unfortunately, as {\tt FLT\_MAX
  - 1}, for instance, is regarded as undefined behavior and will on most
platforms equal to {\tt FLT\_MAX}. Inducing undefined behavior when trying to
deduce whether an instruction is safe, renders it's point moot.

We have therefor chosen to use a simple, but also sub optimal, solution. When,
for instance, adding two floats together that should produce overflow, the
resulting value will become equal to {\tt FLT\_MAX} (in the case of single
precision). We therefor check whether the result of each operation either equals
the maximum or minimum value for the given precision. In cases where an
operation equals exactly the maximum or minimum value, an overflow exception
will be thrown, even though under- or overflow will not have occurred. TODO:
need more comment?


\subsection{Heap Objects}


\subsection{Threading}
% Abstraction
% Structure (tree)
%% Main thread

To support concurrent programming fundamental mechanisms of threading has to be
supported. These include creating and destroying threads while also having
synchronization structures for mutual exclusion, such as mutexes.

We have done this by wrapping existing implementations of threading libraries.
This way we can easily change the threading back-end by choosing which library
wrapper to be compiled with the executable. The interface for threads will
therefor be the same, regardless of which library is handling the logic.

The thread interface is defines in {\tt thread.h}, while the back-ends will live
in {\tt thread\_<lib>.c}.

The abstract machine itself utilizes threading by always running programs
through threads. That way, all errors will safely be handled without crashing
the main process. This is paramount, as machine crashes are considered critical
errors and should never occur.

Threads will be structured in a tree fashion, where a thread can spawn multiple
child threads.
% TODO: update
\begin{ccode}
struct thread_state {
    struct thread_state *parent;
    struct thread_state *children;

    void *thread;

    uint32_t pc;
    stack_t stack;

    ...
};
\end{ccode}

As shown above, each thread state has a void pointer called {\tt thread}
(TODO). This is used by the chosen threading library, where itself needs to
create the necessary structures and track the data. This will likely consist of
some internal thread identifier and possible mutex variables.

\subsection{Meta Information}
%%% Local Variables:
%%% mode: latex
%%% TeX-master: "../report"
%%% End:

It is not only for stack or heap elements which we need to store information
about their type. Sub-routines and object fields also has type information which
we need to keep track of, but the information is not the same. For instance, a
stack or heap element has types describing the type of its value. A sub-routing
will need have a type signature, describing the types of it's parameters, a name
and possible properties. Similarly an object field will have (TODO).

Therefore, we need to abstract our type information a level higher, letting us
better describe different types of types. Instead of storing the different data
different places, we will have a centralized table, containing all meta
information. This will be named the \term{meta table} and will need to be
dynamically sized to support reflection and generating new types at run-time.

All meta objects in the machine will have a tag defining which of type of
information it contains.
\begin{ccode}
enum meta_tag_e {
    TYPE,
    METADATA,
    FIELD,
};
typedef enum meta_tag_e meta_tag;

struct meta_base_s {
    meta_tag tag;
};
typedef struct meta_base_s meta;

struct meta_type_s {
    meta base;

    type *type;
};
typedef struct meta_type_s meta_type;
\end{ccode}
In the case of {\tt TYPE} it will be cast to the {\tt meta\_type} which allows
access to the {\tt type} referenced described below.

We will describe each type of meta information in turn.

\subsubsection{Types}
The executable file, containing the program to be run, declares a type table for
the custom types created by the program. Through out the executable, these types
are referenced through the type's index in this table. This table is static, so
the machine can analyze how many types it contains. The types will be parsed
from the executable and added to the global meta table lazily. This reduces
the initial overhead of parsing the executable.

The machines built-in types are described by a type tag, implemented as an enum:
\begin{ccode}
enum type_tag_e {
    ACTIVATION_ELEMENT = 0,
    ANY                = 1,
    BOOL               = 2,
    INT8               = 3,
    UINT8              = 4,
    ...
};
typedef enum type_tag_e type_tag;
\end{ccode}
The enums integer value is used to describe its index in the meta table. There
will always only be {\it one} instance of each type, so two equal types can
never be referenced by two different pointers. This invariant lets the machine
check type equality through comparing the type pointers.

More specifically, all types in the machine is stored internally through
structs. Simple types implement the {\tt type\_base} struct, which is extended
in composite types. For instance, a reference is represented as:
\begin{ccode}
struct type_base_s {
    type_tag tag;
    int size;
};
typedef struct type_base_s type;

struct type_ref_s {
    type base;
    type *ref_type;
};
typedef struct type_ref_s type_ref;
\end{ccode}

All types are always passed around in the machine through the {\tt type}
name. If the type is for instance a reference, i.e. implementing the {\tt
  type\_ref}, it is cast so its reference specific attributes can be
referenced. The type of type is checked through the tag, defined in {\tt
  type\_base}. Another important entry in the type structure is the size
attribute. This is essential, as it tells the machine how much memory is needed
to store its value on either the stack or heap.

Types from the executable are mapped from its index in the binary type table to
the machines meta table. The types from the binary file is parsed lazily,
i.e. the first time the program tries to look-up a type through its index in the
binary type table, it is parsed and stored in the machine's type table. This is
done in the virtual machine implementation's {\tt vm\_lookup\_elf\_type}
function, which takes a reference to the type table and a type index in the
binary file.

TODO: update look-up shit
To get a reference to a built-in type, one can look-it up through a look-up
function, which takes the name of the built-in type and the type table:
\begin{ccode} % TODO: update
type *lookup_type(type **ttable, type_tag tag)
{
    int i;
    for (i = 0; i < ttable_size; i++) {
        if (ttable[i] == NULL)
            continue;

        if (ttable[i]->tag == tag)
            return ttable[i];
    }

    log_errf(TYPE_ERROR, "could not find type with tag %d", tag);

    return NULL;
}
\end{ccode}
The function iterates the types in the type table, matching on its tag. The null
check is used to avoid types from the binary file not yet parsed, to be skipped.

Currently, if a type is not found, an error is thrown, halting the machine (TODO
exceptions).

When trying to box a stack element trough the {\tt box} instruction, the top
element is popped off the stack and stored on the heap. An element with a
reference to the heap object is in turn pushed to the stack. The type of the new
element will be a {\tt Reference<t>} type, where {\tt t} is the type of the
stack element which was boxed. If this type is not found in the type table
already it will have to be generated. This is done by TODO.

Composite types are made through (TODO).

Most types can be converted, with some exceptions (TODO signature type). If a
value is converted to a type with smaller size, for instance an {\tt Int32} to
{\tt Int8}, the value is truncated (TODO exception)?

All types are garbage collected. This means, when a type is no longer used by
the program, it's memory is freed. (TODO)

% floats
When handling floating point precision numbers we will, as aforementioned, use
the IEEE 754 standard~\cite{ieee754}.
TODO:
\begin{itemize}
  \item General
  \item Special values (NaN, infinity, +/- 0)
  \item Binary representation
  \item binary32 vs binary64
  \item To and from C float/double
\end{itemize}

% type lattice
When performing arithmetic instruction on numbers, the type of the number has to
be taken into careful consideration. There is always the common pitfall of not
handling overflow of numbers, but also when doing arithmetic operations of two
different types of numbers. If a program, for instance, divides a floating point
precision number by an integer number, will the result be an floating point or
integer? Instead of having a long list of each case, the choice will be defined
by a type lattice. This type lattice will control what type the result is of a
given operation between two numbers of different types. TODO:
\begin{itemize}
  \item The type lattice
  \item Handling errors (div by zero, overflow, etc.)
\end{itemize}


\subsection{Scopes}

\subsection{Instructions}
\label{sec:implementation:instr}
\subsubsection{Control Flow}
\label{sec:implementation:instr:control-flow}

At its core control flow is about changing the program counter of an executing
thread. The program counter, also known as the instruction pointer, is an
unsigned integral value representing an address in the byte code where the
machine will start parsing the next instruction. Instructions can change the
value of the program counter during execution which is the basis of all control
flow instructions, including branching and sub-routine calls.

The program counter is implemented as a field on the thread state and since the
thread state is the fundamental argument to all instruction functions, it is
trivial to change the program counter at any time. Because the byte code
programs are byte \textit{addressed} and most instructions span several bytes,
it is possible to set the program counter to a byte that is in the middle of an
instruction, which would lead to unpredictable unwanted behavior. Assemblers
usually use named labels that mark specific locations in assembly code and are
converted to actual addresses when assembled into binary format. It could also
be a task of the byte code verifier to assert that addresses always point to the
beginning of an instruction (as described in
Section~\ref{sec:separate-components:verifier:branch}).

Branching is the most fundamental control flow operation and is implemented
simply by setting the executing thread's program counter directly. Certain
checks are made to verify that a given address is a legal branch target.

Sub-routines are essentially implemented in the exact same way as branching
operations but with the addition of an activation element being pushed to the
stack. The activation element contains the address of the call site which is
used to set the program counter back when a return instruction is executed. In
addition to being on the stack, pointers to activation elements are saved as a
linked list in the thread state making it possible to retrieve its information
without having to search through the stack. Among others, it is used to check
whether the program counter is exceeding the address bounds of the current
sub-routine, which will result in an implicit return.

\subsubsection{Prefixes}

Instructions can be prefixed to modify the behavior of certain instructions.

Integer arguments to instruction are by default 8-bit integer values meaning
that a single code byte is consumes from the byte code stream. The {\tt large}
prefix extends this to 32-bits, i.e.~four bytes, which will likewise be consumed
and the program counter incremented accordingly.

The {\tt noOverflow} prefix signals the machine to throw exceptions if overflow
occurs when executing the following arithmetic instruction. Overflow is detected
differently for the various arithmetic operations and operand types. This is
required as different as types have different maximum and minimum values. We
will briefly explain the challenge of handling overflow in different scenarios
and how \thename{} currently, and imperfectly, tackles them.

Looking at how to detect overflow when doing arithmetic operations on {\em
  signed integer} operands, we have to take the size of the the operands' type
into account. For instance, an {\tt Int8} can only represent integer numbers in
the rage of -128 to 127. Knowing the type's maximum and minimum value, we can
deduce whether the operation will induce over- or underflow on operand $a$ and
$b$. In the case where the operation is addition, we check if operand $b$ is
larger than zero {\em and} operand $a$ is lesser than the sum of the type's
maximum value and $b$'s inverse. This can be denoted in the following formula:

\label{eq:overflow}
\begin{equation}
  overflow = (b > 0) \wedge (a > max - b)
\end{equation}

In a similar fashion we can detect underflow by:

\begin{equation}
  underflow = (b < 0) \wedge (a < min - b)
\end{equation}

If the operation is subtraction, we just change the plus operator to minus, in
the two equations. In the case of multiplication we use division, etc.

For unsigned integers the case is a little simpler, as the operand cannot be
negative. Therefore, when doing addition, underflow can per definition not
occur:

\begin{equation}
  a + b > 0, \qquad when\ a > 0, b > 0
\end{equation}

For substitution, we only needs to check if the $b$ operand is greater
than $a$, which will result in a negative number, thus inducing underflow.

\begin{equation}
  a - b < 0, \qquad when\ a > 0, b > a
\end{equation}

These fairly simple solutions has shown through tests, to work well where the
operand precision is not {\tt Int64} or {\tt UInt64}. The reason for this is
that all overflow check function promote each operand to an {\tt int64\_t} or
{\tt uint64\_t}, which prevents overflow of all operands with lower
precision. In the case of 64-bit integers, there is a risk of the check itself
to overflow. More precisely, $a < min - b$ will overflow if $b$ is
positive. There is not good solution for this, other than to use a higher
precision number for the tests. As C does not have any built-in data types of
larger precision, one would have to use multi-precision numbers which would
require some time to implement. We have chosen to note the short comings of the
current solution and focus our attention else where.

Both GCC and Clang has built-in functions for doing integer arithmetic while
also detecting
overflow\footnote{\url{https://gcc.gnu.org/onlinedocs/gcc/Integer-Overflow-Builtins.html}}\footnote{http://clang.llvm.org/docs/LanguageExtensions.html\#checked-arithmetic-builtins}. GCC's
implementation does this by converting the two operands into infinite precision
numbers, where after doing the the operation on the promoted operands. The
result is then cast to the original precision. If the casted result does not
equal the infinite precision result, overflow has occurred, which is indicated
by the functions return value. We have chosen to not use these to hold the
implementation as library independent as possible.

Detecting overflow for floating-point precision numbers is somewhat more
challenging, due to the more complex nature of its encoding. The C Standard
Library has built-in constants for finding a float's maximum value; {\tt
  FLT\_MIN} and {\tt FLT\_MAX} for single precision, {\tt DBL\_MIN} and {\tt
  DBL\_MAX} for double precision. These could in theory be used in the same
fashion as above, creating an efficient solution, given that we could do
built-in arithmetic operations on them. This is unfortunately not true, as {\tt
  FLT\_MAX - 1}, for instance, is regarded as undefined behavior and will on
most platforms be equal to {\tt FLT\_MAX}. Inducing undefined behavior when
trying to deduce whether an instruction is safe, renders its point
moot.

We have therefore chosen to use a simple but also sub-optimal solution. When,
for instance, adding two floats together that should produce overflow, the
resulting value will become equal to {\tt FLT\_MAX} (in the case of single
precision). We therefore check whether the result of each operation either
equals the maximum or minimum value for the given precision. In cases where an
operation equals exactly the maximum or minimum value, an overflow exception
will be thrown, even though under- or overflow might not have occurred.

\subsubsection{Arithmetic}

The arithmetic operations we have implemented is addition, substitution,
multiplication, division, remainder and negation. With the exception of the
latter, these are all binary operations, used through reverse-polish notation,
or suffix notation. We have made use of C's build in operators to make this as
efficient as possible. As the machine enforces strong typing, an exception will
be thrown if the operands are not of the same type. The only exception is if the
operand is of {\tt AnyType}, where it will automatically be converted to a
matching type.

Different types have to be handled differently to ensure correct
computations. For instance, an integer type cannot be used in the same way as a
floating-point number. To do this, we interpret the operands to its
corresponding C built-in type, perform the operation, and converting them back
to its standard machine type. We cannot rely on GCC to cast the types directly,
as this would be platform dependent; different platform use different endianness
and standards for storing number in memory.

These binary arithmetic operations are very similar in implementation, so we
have chosen to use C macros to keep the code base tidy, and not have to repeat
our selves. In summary this macro does the following:

\begin{enumerate}
  \item Ensure the top two stack elements are of the same type or {\tt AnyType}
  \item Pop them off the stack
  \item Align the types if either is {\tt AnyType}
  \item Allocate space for the result
  \item Decode the operands to their corresponding C type
  \item Perform the arithmetic operation
  \item If the {\tt noOverflow} prefix is set, check for overflow
  \item Encode result back to machine type
  \item Push the result onto the stack
\end{enumerate}

The saturated variants of the arithmetic instructions are implemented by
bounding the result of an arithmetic instruction within the range of the given
type.

As explained in Section~\ref{sec:design:stack:logic}, logical operations include
{\tt and}, {\tt or}, {\tt xor}, {\tt not}, left- and right-wise bit-shift ({\tt
  shl}, and {\tt shr}) and arithmetic right-wise bit-shift ({\tt ashr}). With
the exception of {\tt not}, their corresponding instructions are implemented in
a similar fashion as those of the arithmetic, through a macro using C's built-in
operators. The {\tt not} instruction takes a {\tt Boolean} value, which is
either {\tt 0x00} or {\tt 0x01}, and flips its value.

Lastly, we also have two sets of compare instructions, either comparing two
numeric operands of matching type or comparing a single numeric operand to zero
(or null). Each category includes equals, lesser-than and greater-than
instructions, all of which push a {\tt Boolean} result value onto the stack.

\subsubsection{Boxing}

% TODO: review
As explained in Section~\ref{sec:design:object-model}, all stacks are private to
the thread owning, making the only possible mechanism of sharing data to move
the data to the stack, i.e.~boxing it.

When trying to box a stack element trough the \instr{box} instruction, the top
element is popped off the stack and stored on the heap. An element with a
reference to the heap object is in turn pushed onto the stack. The type of the
new element will be a {\tt Reference<t>} type, where {\tt t} is the type of the
stack element which was boxed. If this type is not found in the meta table
already it is generated at run-time.

The \instr{unbox} instruction essentially does exactly the opposite, pushing the
referenced value to stack and asking telling the garbage collector that its
former heap object is not any longer used.


%%% Local Variables:
%%% mode: latex
%%% TeX-master: "../report"
%%% End:


\subsection{Binary file}

\subsection{Analysis}
% Valgrind
% gprof?

\subsection{Testing}
Throughout the project we have made heavy use of testing to ensure the
correctness of the implementation, i.e.~whether it strictly follows the machine
specification~\ref{sec:spec}. We have both made use of \term{unit}- and
\term{integration-tests}. Here unit-tests have the purpose of testing a single
isolated component, typically a single function. We mock a specific state of the
machine, or the state of a component in the machine, and then run isolated
functions where we {\it know} what the result should be, which can easily be
asserted. By comparison, the integration-tests have a much larger scope, in that
their purpose is to test the whole machine without any regard to any specific
component. This effectively ensures each component is wired together correctly.

For parts of the implementation we have made use of \term{test-driven
  development} (TDD). We write the tests prior to implementing the actual
functionality being tested. This eliminates the danger of false positives, where
tests pass when they are not supposed to. It also offers an efficient work flow,
where the minimum functionality is implemented to make the tests pass.
%TODO det har vi ikke sådan RIGTIG, evt w/e

\subsubsection{Unit-tests}

Unit-testing in C is a fairly simple process and essentially does not require
any framework or library. While not required, a simple set of \term{macros} will
greatly improve the readability and ease of writing tests.

% cmocka
In spite of this, we have chosen to use a unit-testing library called {\it
  Cmocka}\cite{cmocka}. It is a well tested and documented library, also used by
large projects like {\it libssh}~\footnote{Implementation of the SSH protocol:
  \url{http://libssh.org/}}. It offers several features which makes unit-testing
more powerful and simpler to write. For instance, it offers test suites and
\term{mocking} of objects which enabled us to set up a specific state of the
machine. With this state, we can run isolated tests, manipulating the state and
thereafter asserting that the correct transformations and output has occurred.

Cmocka also allows the test program to recover from signaled exceptions,
e.g. {\tt SIGSEGV}, {\tt SIGKILL}, etc. If a test in the test program triggers a
segmentation fault exception, for instance, it will not exit, but rather show
where the exception occurred and print useful debugging information like the
call stack.

Lastly, the library works on a wide range of platforms and only depends on the C
standard library. This makes it possible to use the library on embedded
platforms and with different compilers.

% example
As an example, we will describe how we test parts of the stack implementation.

Firstly, we create a mocked state of a stack which we can use for our tests.

%XXX assertions shouldn't happen in setup
\begin{lstlisting}[language={[ANSI]C},caption={Unit-test setup procedure}]
  static int setup(void **state)
  {
    stack_t *s = malloc(sizeof(stack_t));
    stack_init(s, 100);

    assert_non_null(s->elements);
    assert_int_equal(100, s->max_size);
    assert_int_equal(0, sum_stack(s));

    *state = s;

    return 0;
  }
\end{lstlisting}

In the above code listing, we initialize a stack object, do some simple
assertions, and store it in the {\tt state} variable passed with the setup
function. As we see below, this state is given as argument to all test cases
which easily allows us to retrieve it by dereferencing.

\begin{lstlisting}[language={[ANSI]C},caption={Unit-test of {\tt stack\_pop}}]
  static void test_pop(void **state)
  {
    stack_t *s = *state;

    uint32_t sum = sum_stack(s);

    stack_push(s, make_se_int(1));
    stack_push(s, make_se_int(2));
    stack_push(s, make_se_int(3));

    assert_int_equal(3, SE_INT(stack_pop(s)));
    assert_int_equal(2, SE_INT(stack_pop(s)));
    assert_int_equal(1, SE_INT(stack_pop(s)));

    assert_int_equal(sum, sum_stack(s));
  }
\end{lstlisting}

After having retrieved the stack for the {\tt state} parameter, we create an XOR
sum of the stack. When having tested the specific function, we assert that the
new XOR sum is the same as before, making sure that there is no unexpected
changes to the stack. Due to the simple nature of our {\tt stack\_sum} function,
this is not a guarantee, as different states of the stack may compute the same
sum, but will however catch most such problems.

In the body of a test case, the {\tt stack\_pop} function in this case, we push
several elements to the stack and assert that they are popped off again in the
correct order.

After all test cases has completed, a teardown function is run, reversing that
of the setup function. In the case of the stack tests, it simply frees the
memory allocated to the state of the stack.

\subsubsection{Integration-tests}

We want our integration tests to test the implementation in the fashion of
black-box testing, i.e.~given some input, the program should produce a specific
output, regardless of how the program works internally. The easiest way of
accomplishing this is by taking the actual machine executable and running it
with certain programs and arguments. We have therefor created several different
programs, from which we know the desired output.

As an example we can check that the machine fails when stack underflow occurs,
i.e.\ popping more values off the stack then there are actual values.

A valid test program for this would consist of a single instruction, \instr{POP,
  1}, producing a stack underflow error. We can test the output of the machine
by both asserting the exit code of the process, where we will have specified the
meanings of different exit codes, and by asserting what is written to standard
out and error ({\tt stdout} and {\tt stderr}). In the case of stack underflow
the exit code should be 4 and the output should match ``underflow'' (TODO:
update).

To automate this process we use shell scripting. For convenience, we have chosen
to utilize a shell testing framework, called shUnit2\footnote{shUnit2, unit
  testing for shell scripts: \url{http://code.google.com/p/shunit2/}}. This lets
us streamline the integration testing process, in contrast to manually running
each machine program and checking the output.

shUnit2 enables us to have an alias for the actual machine binary, letting us
set the target binary through our build system and piping standard error to
standard out. In our test cases we can then, by using the alias, store the
output and exit code in variables and do assertions based on their expected
values.
% TODO: update return code
\begin{lstlisting}[language={sh},caption={shUnit2 underflow test case}]
  test_stack_underflow()
  {
    output=$(am --file ${PDIR}/underflow.amb)
    ret=$?

    assertEquals "return code should be 4" 4 $ret
    assertTrue "output should include underflow" \
    "contains \"$output\" \"underflow\""
  }
\end{lstlisting}

In the above test case we run the machine with the underflow program, asserting
that the process exits with the correct code and that the output contains the
word `underflow`.


% Assembler
%% Simple parsing/emitting
%% Label calculations (offsets)

% Object model
%% Virtual tables
%% (Un)boxing

% Execution model
%% Single stack
%% Return values as output arguments
%% No frame pointer(?)

% Binary format (ELF)
%% Sections
%% Type encoding
%% Instructions
%%% Argument encoding

% Exception handling

% Debugging information

%%% Local Variables:
%%% mode: latex
%%% TeX-master: "../report"
%%% End:
