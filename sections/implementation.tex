\subsection{Tools and Languages}

We have implemented \thename{} in the C programming language, more specifically
using the C99 standard with GNU extensions by use of GCC\footnote{The GNU
  Compiler Collection: \url{https://gcc.gnu.org}}. There are both pros and cons
of using a relatively low-level language (as compared to other available
languages), but we found that the benefits outweighed the disadvantages.

First of all C is extremely portable; GCC is available on \textit{a lot} of
architectures, and all modern operating systems ship with a C compiler which
essentially is all that is required to build \thename{} for a new platform. The
standard libraries and architecture specific functions available differ with
each machine and device, so we aim at relying on as few external libraries as
possible and provide wrappers for the ones we use so that the implementation can
be ported quickly.

Secondly we do not want to rely on a garbage collector provided by languages
like C++ and Java. We need to full control over where each bit of data is
stored, mostly for performance reasons both in terms of memory footprint and
execution speed.

Third, speed is of the essence and C is notoriously fast, given that the code is
written in a good manner.

There are however downsides to using C. Because we are not veterans of C and due
to its low-level nature it is inherently more complex to develop features that
could be implemented in a few lines of code in a high-level language. Memory has
to be tracked meticulously to prevent memory leaks. We have found that it is a
great exercise for becoming proficient with programming in general because it
provides us with knowledge of the mechanisms that lie behind concepts that are
generally taken for granted.

For debugging we use gdb via GNU Emacs\footnote{GNU Emacs:
  \url{http://www.gnu.org/software/emacs}} and for analyzing memory usage and
fixing memory leaks we use the Valgrind instrumentation tool\footnote{Valgrind:
  \url{http://valgrind.org}}.

The code is extensively documented using Doxygen\footnote{Doxygen:
  \url{http://www.stack.nl/~dimitri/doxygen}} style comments.

\textbf{Note:} The code snippets presented in the following sections does not
always correspond directly with the code found in the actual
implementation. They carry the gist of what we want to communicate but generally
avoids unnecessary and redundant information not relevant to the individual
cases.

\subsection{Core Infrastructure}

The infrastructure of \thename{} is the code that lays the basis for interaction
between the individual parts of the system. It is a collection of machinery
which hand control back and forth between each other, such as the instruction
cycle, thread management and exception handling. Besides the machine's exposed
exception handling system there is an internal error handling mechanism which
allows errors within the machine to be handled as gracefully.

\subsubsection{Instruction Cycle}

The fundamental means of operation in the machine is the execution of
instructions parsed and interpreted from byte code. Each thread runs their own
cycle, the essence of which is shown in
Listing~\ref{lst:implementation:instruction-cycle}

\begin{lstlisting}[%
  caption={Pseudo representation of the instruction cycle},%
  label={lst:implementation:instruction-cycle}]
while (machine is running) {
  fetch opcode
  fetch potential arguments
  execute instruction corresponding to opcode
  update program counter
}
\end{lstlisting}


% vm initialization
% instruction cycle
% memory layout (stack per thread, heap area, code area)
% thread

% Very general architecture (theres a vm, a file reader, instruction codes, etc)
% What's in the VM state and Thread state?
% How code bytes are handled
% Instruction cycles
% Internal error handling
% Logging/debugging facilities
% Graceful exits / clean up
% Command line arguments

\subsection{Stacks}
A stack is in itself a fairly simple construction that allows values to be
pushed to an popped from an allocated memory region. Stack elements are stored
and retrieved by the Last-In, First-Out (LIFO) principle.

\subsubsection{Stack elements}

Each element on a stack holds information about both the \textit{declared} and
\textit{actual} type of data that it contains. Details on this is described in
\ref{NEEDED}. The data part of a stack element is a simple array of bytes,
stored in big endian order, i.e.~the most significant bit is at the lowest
memory address.

\subsubsection{Memory Layout}

Arguably the most intuitive way to construct a stack is to use a simple array of
a fixed initial size and handle operations by keeping track of the number of
elements in the stack. A push operation would thus increments the number of
elements and assigns a value to the new top-most slot. A pop operation will
correspondingly decrement the number of elements in the stack and potentially
return the element that is being removed. When the number of elements in the
stack exceeds the initial size of the containing array more memory must be
allocated. This can be done simply by use of \code{realloc} where the common
approach is to double the amount of allocated memory, to adapt to quickly
growing stacks. However, \code{realloc} can be a performance issue because it
will create a new allocation, copy over the current data and free the old
allocation, in the case the there is no more space directly following what is
being reallocated~\cite{man-realloc}. This is arguably an expensive operation,
and since the stack is by far the most frequently used means of operation in
\thename{}, it is critical that it performs well.
% TODO: agree ^ realloc not _always_ moves memory, but etc?

An important thing to consider is that the content of a stack element depends on
the type of the element, and thus have varying sizes. An element whose type is
\code{Int8} requires a single byte for storage while an element of type
\code{Int32} requires 4 bytes of storage and static arrays can be arbitrarily
large. Since we do not want to waste memory on redundant bytes, that prevents us
from implementing stack elements as fixed size structures laid out in an array
as described above. We have solved the problem by implementing stack elements as
a singly linked list going from the top element down to the bottom. The memory
used by the stack should be as contiguous as possible, to avoid a large amount
of allocations and deallocations, and the elements in the linked list should
reside one after another in memory. The actual content of an element is stored
\textit{in between} the linked list nodes, and the \code{next} pointer simply
points to the memory address following its content data
bytes. Figure~\ref{fig:implementation:stack-layout} shows how the stack elements
are laid out, with arrows denoting pointers.

\begin{figure}[H]
  \centering
  \begin{drawstack}[every initial by arrow/.style={*->}]
  \startframe{}
  \cell{Declared type} \cellcom{\texttt{rtti*}} \coordinate (next) at (currentcell.east);
  \cell{Actual type} \cellcom{\texttt{rtti*}}
  \cell{Next} \cellcom{\texttt{stack\_element*}}
  \cell{Content size} \cellcom{\texttt{int}}
  \padding{2}{Content}  \cellcom{\texttt{char[]}}
  \finishframe{\texttt{stack\_element}}

  \startframe{}
  \cell{Declared type}  \cellcom{\texttt{rtti*}}
  \cell{Actual type}  \cellcom{\texttt{rtti*}}
  \cell{Next} \cellcom{\texttt{stack\_element*}} \coordinate (el2) at (currentcell.east);
  \cell{Content size} \cellcom{\texttt{int}}
  \padding{2}{Content}  \cellcom{\texttt{char[]}}
  \finishframe{\texttt{stack\_element}}
  \startframe{}

  \draw[->, thick, bend right=20] (el2) edge (next);
\end{drawstack}

% \begin{drawstack}
%   \startframe
%   %   \cellcom writes something on the right-hand side of a cell.
%   \cell{loc2} \cellcom{-8(\%ebp)}
%   \cell{loc1} \cellcom{-4(\%ebp)}
%   %   \esp and \ebp are stack pointer and base pointer in Pentium.
%   %   These macros are simple shortcuts for \cellptr{...}
%   \cell{Sauvegarde \%ebp} \esp \ebp
%   \cell{@ retour} \cellcom{4(\%ebp)}
%   \finishframe{fonction\\ {\tt f}}
%   \startframe
%   \cell{} \cellcom{8(\%ebp)}
%   \cell{}
%   \finishframe{fonction\\ {\tt main}}
% \end{drawstack}

% \section{Padding}

% \begin{drawstack}
%   \cell{above padding}
%   \padding{3}{nothing here}
%   \cell{below padding}
% \end{drawstack}

% \section{Below/Above stack pointer}

% \begin{drawstack}
%   \cell{Top}
%   \cell{Below top}
%   %   \bcell is just like \cell, but in a different color.
%   \bcell{Above bottom} \cellptr{Stack pointer here}
%   \bcell{Bottom}
% \end{drawstack}

% \section{Highlighting some cell}

% \begin{drawstack}
%   \cell{Uninteresting cell}
%   \cell{Interesting cell} \cellround{Yes, this one!}
% \end{drawstack}

% \section{Structures without a stack structure}

% \begin{tikzpicture}
%   \draw (3, -1) node (Otm) {
%   \begin{tabular}{c}
      %       Object\\vtable
      %     \end{tabular}
      %       };

      %       \drawstruct{(0,0)}
      %       \structcell[freecell]{~} \coordinate (Atm) at (currentcell.east);
      %       \structcell[freecell]{\texttt{@Object.equals()}}
      %       \structcell[freecell]{\texttt{@code A.m()}}
      %       \structcell[freecell]{\texttt{@code A.p()}} \coordinate (A) at (currentcell.west);
      %       \structname{
      %       \begin{tabular}{c}
      %       A's vtable
      %     \end{tabular}
      %       }

      %       \drawstruct{(-4,-3)}
      %       \structcell[freecell]{} \coordinate (Btm) at (currentcell.east);
      %       \structcell[freecell]{\texttt{@Object.equals()}}
      %       \structcell[freecell]{\texttt{@code A.m()}}
      %       \structcell[freecell]{\texttt{@code B.p()}}
      %       \structcell[freecell]{\texttt{@code B.q()}}
      %       \structname{B's vtable}

      %       \draw[->] (Btm) -- (A);
      %       \draw[->] (Atm) -- (Otm);
      %       \end{tikzpicture}

      %       \section{Structures and stack together}

      %       \begin{tikzpicture}[scale=.8]

      %       \stacktop{}
      %       \separator
      %       \cell{\texttt{p3}}        \cellcomL{11(GB)} \coordinate (p3) at (currentcell.east);
      %       \separator
      %       \cell{\texttt{p2}}        \cellcomL{10(GB)} \coordinate (p2) at (currentcell.east);
      %       \separator
      %       \cell{\texttt{p1}}        \cellcomL{ 9(GB)} \coordinate (p1) at (currentcell.east);
      %       \separator
      %       \cell{\texttt{@P3D.diag}} \cellcomL{ 8(GB)}
      %       \cell{\texttt{\footnotesize @Object.equals}} \cellcomL{ 7(GB)}
      %       \cell{\texttt{3(GB)}}     \cellcomL{ 6(GB)} \coordinate (T1) at (currentcell.east);
      %       \separator
      %       \cell{\texttt{@P2D.diag}} \cellcomL{ 5(GB)}
      %       \cell{\texttt{\footnotesize @Object.equals}} \cellcomL{ 4(GB)}
      %       \cell{\texttt{1(GB)}}     \cellcomL{ 3(GB)} \coordinate (T2) at (currentcell.east);
      %       \separator
      %       \cell{\texttt{\footnotesize @Object.equals}} \cellcomL{ 2(GB)}
      %       \cell{\texttt{null}}      \cellcomL{ 1(GB)}
      %       \cell[draw=none]{Stack}


      %       \drawstruct{(5,1)})
      %       \structcell{z=2,5}
      %       \structcell{y=2,5}
      %       \structcell{x=2,5}
      %       \structcell{.} \coordinate (O1) at (currentcell.west);
      %       \coordinate (O1l) at (currentcell.south);

      %       \drawstruct{(9,-3)}
      %       \structcell{y=1}
      %       \structcell{x=1}
      %       \structcell{.} \coordinate (O2) at (currentcell.west);
      %       \coordinate (O2l) at (currentcell.south);

      %       \draw[->] (p3) -- (O1);
      %       \draw[->] (p2) -- (O1);
      %       \draw[->] (p1) -- (O2);

      %       \draw[->] (O1l) .. controls (O1 |- T1) .. (T1);
      %       \draw[->] (O2l) .. controls (O2 |- T2) .. (T2);

      %       \draw (10,-10) node{Heap};

      %       \end{tikzpicture}



      %       \section{Using tikzpicture instead of drawstack}

      % %       The environment drawstack is basically a syntactic sugar for
      % %
      % %       \begin{tikzpicture}[#1]
      % %         \stacktop{}
      % %         ...
      % %         \stackbottom
      % %       \end{tikzpicture}
      % %
      % %       You can use the above syntax for more flexibility.

      %       \begin{tikzpicture}[scale=0.8]
      %       \small
      %       \stacktop{}
      %       \cell{My cell}
      %       \stackbottom{}
      %       \end{tikzpicture}

      %       \section{Changing style}

      %       {% tikzstyle will be local to this {...}
      %       \tikzstyle{freecell}=[fill=blue!10,draw=blue!30!black]
      %       \tikzstyle{occupiedcell}=[fill=blue!10!orange!10,draw=blue!30!black]
      %       \tikzstyle{padding}=[fill=yellow!20,draw=blue!30!black]
      %       \tikzstyle{highlight}=[draw=orange!50!black,text=orange!50!black]

      %       \begin{drawstack}
      %       \cell{Uninteresting cell}
      %       \cell{Interesting cell} \cellround{Yes, this one!}
      %       \bcell{bcell}
      %       \padding{2}{Padding}
      %       \end{drawstack}
      %       }

      %       \section{Example: Computing Factorial}

      %       \begin{drawstack}[scale=0.8]
      %       \startframe
      %       \cell{N = 1}
      %       \cell{...}
      %       \finishframe{fact(1)}
      %       \startframe
      %       \cell{N = 2}
      %       \cell{...}
      %       \finishframe{fact(2)}
      %       \cell{$\vdots$}
      %       \startframe
      %       \cell{N = 5}
      %       \cell{...}
      %       \finishframe{fact(5)}
      %       \end{drawstack}

  \caption{Stack memory layout}
  \label{fig:implementation:stack-layout}
\end{figure}

The C language does not support variable length arrays, which means that an
array must have its length declared, otherwise the type is regarded as
incomplete. However, line~\ref{code:stack-element:flexible-data} in
Listing~\ref{lst:implementation:stack:element} shows a feature known as a
\term{flexible array} that allows the last field of a struct to be declared as a
size-less array. We take advantage of exactly that so that we can easily
interpret the \code{data} field of an element as an array of bytes of any size.

What is left to solve is the problem of a stack growing beyond its initial
size. As mentioned, reallocation of the memory used by the stack works, but
results in poor performance. To mitigate the issue we have modelled the stack as
a linked list of allocated memory regions which we call \textit{segments}. Each
segment is essentially a stack in itself and is implemented exactly as described
above. The benefit is that when a segment becomes full, a new segment is simply
allocated and wired into the linked list, thus preventing the need for a
reallocation. The exposed interface of the stack consists of a set of stack
operation functions that essentially maps operations onto segments by selecting
the appropriate segment, and potentially what index of its elements that will be
the target of a given operation. Thus the segments are hidden in the
implementation and are solely internal data structures which users of the stack
are ignorant of.

\begin{remark}
  A segmented stack is not a novel idea. They are used by the Go programming
  language to facilitate large amounts of unbounded stacks (for concurrent
  sub-routines called goroutines). The Rust programming language did as well but
  later abandoned them. There are arguments for not using them because of the
  overhead generated by extra work needed when stack limits are
  exceeded\cite{rust:segmented-stack, go:segmented-stack}.
\end{remark}

Segments are implemented as a doubly linked list as shown in
listing~\ref{lst:implementation:stack:segment}. Each segment keeps track of its
length (i.e.~number of elements current in the segment) and the top element. The
\code{content} field is the region allocated for storage of elements. The
\code{data\_cursor} is maintained to always point to the byte following the top
element's content bytes and is useful when pushing a new element. The
\code{size} field simply represents the size of the memory region allocated for
the segment and is used to check whether the segment has room for more elements.

\begin{figure}[h]
  \centering
  \begin{lstlisting}[language={[ANSI]C},%
    caption={Structure defining the stack},%
    label={lst:implementation:stack:element}]
struct stack_element {
    /* points to next element in the upward direction */
    struct stack_element_s *next;

    int actual_type;
    int declared_type;

    byte data[]; (*@\label{code:stack-element:flexible-data}@*)
}
  \end{lstlisting}
\end{figure}

A segment has a limited amount of space for elements but the stack itself is
unbounded as a result of the segment list; there is no limit to how many
segments can be allocated, except the amount of physical or virtual memory
available to the machine.

Stack elements can be arbitrarily large. For instance fixed size arrays stored
on the stack are as large as their type indicates, and might be bigger than the
defined segment size. Thus, before pushing an element (as shown in
Listing~\ref{lst:implementation:stack:operations}) we check whether the element
can fit into a segment, and if not a segment is allocated with the size
explicitly given. The new segment is allocated the required size for the element
plus the default segment size as to allow for more elements to be pushed and
minimize the segment handling overhead.

\subsubsection{Caching Stack Segments}

A new segment is allocated when the last active segment becomes full, but is not
immediately freed when it becomes empty. Instead a number of segments are cached
to facilitate series of stack operations that operate right between the limit of
two segments. Consider a loop of code that repeatedly pushes four elements to
the stack only to pop them off moments after. If there are two slots left in the
last segment when the loop begins, a new segment must be allocated during the
course of one loop iteration. However when the iteration pops the values off
again the new segment will become empty, which without the cached segments would
result in the segment being freed. With the cached segments the newly allocated
segment is saved and is simply rewired into the stack when it is needed again.

When there are more inactive segments than defined by
\code{STACK\_NUM\_CACHED\_SEGMENTS} a segment will be freed. This is a trivial
operation because the all of the segment's data and elements' data lie within
the segment's memory allocation, so it is simply freed and there is no need to
free individual elements.

%%% Local Variables:
%%% mode: latex
%%% TeX-master: "../report"
%%% End:


\subsection{Execution Model}


\subsection{Heap Objects}

\begin{figure}[H]
  \centering
  \begin{tikzpicture}
  \draw (9, -1) node (b_vtbl) {
    \begin{tabular}{c}
      Bootstrapped\\vtable
    \end{tabular}
  };

  \drawstruct{(5,0)}
  \structcell[freecell]{Virtual table} \coordinate (vtbl_e) at (currentcell.east); \coordinate (vtbl_w) at (currentcell.west);
  \structcell[freecell]{Type information}
  \structcell[freecell]{Lookup table}
  \structname{Virtual table}

  \drawstruct{(5,-5)}
  \structcell[freecell]{KV1} \coordinate (hm) at (currentcell.west);
  \structcell[freecell]{KV2}
  \structname{Hash map}

  \drawstruct{(0,0)}
  \structcell[freecell]{Virtual table} \coordinate (ho_vtbl) at (currentcell.east);
  \structcell[freecell]{Hash map} \coordinate (ho_hm) at (currentcell.east);
  \structcell[padding]{Object data}
  \structname{Heap object}

  \draw[->, thick] (ho_vtbl) -- (vtbl_w);
  \draw[->, thick] (ho_hm) -- (hm);
  \draw[->, thick] (vtbl_e) -- (b_vtbl);
\end{tikzpicture}
  \caption{Heap object memory layout}
\end{figure}


\subsection{Threading}
% Abstraction
% Structure (tree)
%% Main thread

To support concurrent programming fundamental mechanisms of threading has to be
supported. These include creating and destroying threads while also having
synchronization structures for mutual exclusion, such as mutexes.

We have done this by wrapping existing implementations of threading libraries.
This way we can easily change the threading back-end by choosing which library
wrapper to be compiled with the executable. The interface for threads will
therefor be the same, regardless of which library is handling the logic.

The thread interface is defines in {\tt thread.h}, while the back-ends will live
in {\tt thread\_<lib>.c}.

The abstract machine itself utilizes threading by always running programs
through threads. That way, all errors will safely be handled without crashing
the main process. This is paramount, as machine crashes are considered critical
errors and should never occur.

Threads will be structured in a tree fashion, where a thread can spawn multiple
child threads.
% TODO: update
\begin{ccode}
struct thread_state {
    struct thread_state *parent;
    struct thread_state *children;

    void *thread;

    uint32_t pc;
    stack_t stack;

    ...
};
\end{ccode}

As shown above, each thread state has a void pointer called {\tt thread}
(TODO). This is used by the chosen threading library, where itself needs to
create the necessary structures and track the data. This will likely consist of
some internal thread identifier and possible mutex variables.


\subsection{Types}
The binary file, containing the program to be run, declares a type table for the
custom types created by the program. Through out the binary file, these types
are referenced through the type's index in this table. This table is static, so
the machine can analyze how many types it contains. The machine has it's own
type table, which is a statically sized array. The size is the sum of built-in
types and types from the binary file.This is called the machines \term{type
  table}. It holds references to {\it all} types used in the program.

The machines built-in types are described by a type tag, implemented as an enum:
\begin{ccode}
enum type_tag_e {
    ACTIVATION_ELEMENT = 0,
    ANY                = 1,
    BOOL               = 2,
    INT8               = 3,
    UINT8              = 4,
    ...
};
typedef enum type_tag_e type_tag;
\end{ccode}
The enums integer value is used to describe its index in the type table. There
will always only be {\it one} instance of each type, so two equal types can
never be referenced by two different pointer. This invariant lets the machine
check type equality through comparing the type pointers.

More specifically, all types in the machine is stored internally through
structs. Simple types implement the {\tt type\_base} struct, which is extended
in composite types. For instance, a reference type would be represented as:
\begin{ccode}
struct type_base {
    type_tag tag;
    int size;
};

struct type_ref {
    struct type_base base;
    struct type_base *ptr_type;
};

typedef type_base type;
\end{ccode}

All types are always passed around in the machine through the {\tt type}
name. If the type is for instance a reference, i.e. implementing the {\tt
  type\_ref}, is is cast so its reference specific attributes can be
referenced. The type of type is checked through the tag, defined in {\tt
  type\_base}. Another important entry in the type structure is the size
attribute. This is essential, as it tells the machine how much memory is needed
to store its value on either the stack or heap.

Types from the binary file are mapped from its index in the binary type table to
the machines type table. The types from the binary file is parsed lazily,
i.e. the first time the program tries to look-up a type through its index in the
binary type table, it is parsed and stored in the machine's type table. This is
done in the virtual machine implementation's {\tt vm\_lookup\_elf\_type}
function, which takes a reference to the type table and a type index in the
binary file.

To get a reference to a built-in type, one can look-it up through a look-up
function, which takes the name of the built-in type and the type table:
\begin{ccode} % TODO: update
type *lookup_type(type **ttable, type_tag tag)
{
    int i;
    for (i = 0; i < ttable_size; i++) {
        if (ttable[i] == NULL)
            continue;

        if (ttable[i]->tag == tag)
            return ttable[i];
    }

    log_errf(TYPE_ERROR, "could not find type with tag %d", tag);

    return NULL;
}
\end{ccode}
The function iterates the types in the type table, matching on its tag. The null
check is used to avoid types from the binary file not yet parsed, to be skipped.

Currently, if a type is not found, an error is thrown, halting the machine (TODO
exceptions).

Composite types are made through (TODO).

Most types can be converted, with some exceptions (TODO signature type). If a
value is converted to a type with smaller size, for instance an {\tt Int32} to
{\tt Int8}, the value is truncated (TODO exception)?

All types are garbage collected. This means, when a type is no longer used by
the program, it's memory is freed. (TODO)


\subsection{Binary file}

% Assembler
%% Simple parsing/emitting
%% Label calculations (offsets)

% Stack based
%% Elements are (type, value) pairs
%% Dynamically sized elements
%%% Linked list

% Object model
%% Virtual tables
%% (Un)boxing

% Execution model
%% Single stack
%% Return values as output arguments
%% No frame pointer(?)

% Binary format (ELF)
%% Sections
%% Type encoding
%% Instructions
%%% Argument encoding

% Exception handling

% Debugging information

\subsection{Analysis}
% Valgrind
% gprof?

\subsection{Testing}

% Test-Driven Development (not strictly)
% unit tests
% black box testing
% def. terms

Throughout the project we have made heavy use of testing to ensure the
correctness of the implementation, i.e. whether it strictly follows the machine
specification~\ref{sec:spec}. We have both made use of \term{unit}- and
\term{integration-tests}. Here unit-tests have the purpose of testing a single
isolated component, typically a function. Here we mock a specific state of the
machine, or the state of a component in the machine, and then run isolated
functions where we {\it know} what the result should be, for which we can easily
test. By comparison, the integration-tests have a much larger scope. Their
purpose is to test the whole program without any regard to any specific
component. This effectively ensures each component is wired together correctly.

For parts of the implementation we have made use of \term{test-driven
  development} (TDD). Here we write the tests prior to implementing the actual
functionality being tested. This eliminates the danger of false positives, where
tests pass when they are not supposed to. It also offers an efficient work flow,
where the minimum functionality is implemented to make the tests pass.

\subsubsection{Unit-tests}
Unit-testing in C is a fairly simple process and essentially does not require
any framework or library. While not required, a simple set of \term{macros} will
greatly improve the readability and ease of writing tests.

% cmocka
In spite of this, we have chosen to use a unit-testing library called {\it
  cmocka}~\cite{cmocka}. It is a well tested and documented library, also used
by large projects like {\it libssh}~\footnote{Implementation of the SSH protocol:
  \url{http://libssh.org/}}. It offers several features which makes unit-testing
more powerful and simpler to write. For instance, it offers test suites and
\term{mocking} of objects which enabled us to set up a specific state of the
machine. With this state, we can run isolated tests, manipulating the state and
thereafter checking that the correct transformations and output has occurred.

Cmocka also allows the test program to recover from signaled exceptions,
e.g. {\tt SIGSEGV}, {\tt SIGILL}, etc. If a test in the test program triggers a
segmentation fault exception, for instance, it will not exit, but rather show
where the exception occurred and print useful debugging information like the
call stack.

Lastly, the library works on a wide range of platform and only requires the use
of the C library. This makes it possible to use the library on embedded
platforms and with different compilers.

% example
As an example, we will describe how we test parts of the stack implementation.

Firstly, we create a mocked state of a stack which we can use for our tests.
\begin{lstlisting}[language={[ANSI]C},caption={Unit-test setup procedure}]
  static int setup(void **state)
  {
    stack_t *s = malloc(sizeof(stack_t));
    stack_init(s, 100);

    assert_non_null(s->elements);
    assert_int_equal(100, s->max_size);
    assert_int_equal(0, sum_stack(s));

    *state = s;

    return 0;
  }
\end{lstlisting}

In the above code listing, we initialize a stack object, do some
simple assertions, and store it in the {\tt state} variable passed
with the setup function. As we see below, this state is given as
argument to all test cases which easily allows us to retrieve it by
dereferencing.
\begin{lstlisting}[language={[ANSI]C},caption={Unit-test of {\tt stack\_pop}}]
  static void test_pop(void **state)
  {
    stack_t *s = *state;

    uint32_t sum = sum_stack(s);

    stack_push(s, make_se_int(1));
    stack_push(s, make_se_int(2));
    stack_push(s, make_se_int(3));

    assert_int_equal(3, SE_INT(stack_pop(s)));
    assert_int_equal(2, SE_INT(stack_pop(s)));
    assert_int_equal(1, SE_INT(stack_pop(s)));

    assert_int_equal(sum, sum_stack(s));
  }
\end{lstlisting}

After having retrieved the stack for the {\tt state} parameter, we create an XOR
sum of the stack. When having tested the specific function, we assert that the
new XOR sum is the same as before, making sure that there is no unexpected
changes to the stack. Due to the simple nature of our {\tt stack\_sum} function,
this is not a guarantee, as different states of the stack may compute the same
sum. % TODO: more on stack sum?
In the body of our test case, the {\tt stack\_pop} function in this case, we
push several elements to the stack and assert that they are popped off again in
the correct order.

After all test cases has completed, a teardown function is run, reversing that
of the setup function. In the case of the stack tests, it simply frees the
memory allocated to the state of the stack.
% \begin{lstlisting}[language={[ANSI]C},caption={Unit-test teardown procedure}]
%   static int teardown(void **state)
%   {
%   stack_t *s = *state;
%   stack_free(s);
%
%   return 0;
% }
% \end{lstlisting}


\subsubsection{Integration-tests}
We want our integration tests to test the implementation in the fashion of
black-box testing, i.e. given some input, the program should produce a specific
output, regardless of how the program works internally. The easiest way of
accomplishing this is by taking the actual machine executable and running it
with certain programs and arguments. We have therefor created several different
programs, from which we know the desired output.

As an example we can check that the machine fails when stack underflow occurs,
i.e. popping more values off the stack then there are actual values. A program
doing exactly this could be:
% TODO: update with actual
\begin{lstlisting}[language={bytecode},caption={Machine program producing
    stack underflow}]
  fn main
  {
    pop
  }
\end{lstlisting}

The above machine program will try to pop an element off the stack without there
being any elements, producing a stack underflow error. We can test the output of
the machine by both asserting the exit code of the process, where we will have
specified the meanings of different exit codes, and by asserting what is written
to standard out and error ({\tt stdout} and {\tt stderr}). In the case of stack
underflow the exit code should be 4 and the output should match `underflow`
(TODO: update).

To automate this process we use shell scripting. For convenience, we have chosen
to utilize a shell testing framework (that's right), called {\tt
  shUnit2}~\footnote{shUnit2, unit testing for shell scripts:
  \url{http://code.google.com/p/shunit2/}}. This lets us streamline the
integration testing process, in contrast of manually running each machine
program and checking the output.

shUnit2 enables us to have an alias for the actual machine binary, letting us
set the target binary through our build system and piping standard error to
standard out. In our test cases we can then, by using the alias, store the
output and exit code in variables and do assertions based on their expected
values.
% TODO: update return code
\begin{lstlisting}[language={sh},caption={shUnit2 underflow test case}]
  test_stack_underflow()
  {
    output=$(am --file ${PDIR}/underflow.amb)
    ret=$?

    assertEquals "return code should be 4" 4 $ret
    assertTrue "output should include underflow" \
    "contains \"$output\" \"underflow\""
  }
\end{lstlisting}

In the above test case we run the machine with the underflow program, asserting
that the process exits with the correct code and that the output contains the
word `underflow`.

%%% Local Variables:
%%% mode: latex
%%% TeX-master: "../report"
%%% End:
