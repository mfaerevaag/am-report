In this section we present the knowledge on which the rest of the thesis is
based. We give a brief overview of modern computer architectures and then start
to dissect the subject of abstract machines in more detail. Finally we present
some of the existing solutions available today that have inspired much of the
work presented in this report.

\subsection{Computer Architectures}
\label{sec:background:computer-architectures}

An architecture is what describes the organization and programming model of a
computer including the functionality provided by, and capabilities of, a
computer system\cite{clements06}. The architecture does not dictate how
individual system are constructed neither software- or hardware-wise. Virtually
all modern general-purpose computer systems are based on the von Neumann
architecture originating from papers by John von Neumann in 1945\cite{riley87},
in which he describes the design of a computing machine using now well-known
concepts such as the arithmetic unit (nowadays part of the CPU), control unit,
memory, and input/output. The structure of modern hardware machines is still
surprisingly similar to this design, even though they are vastly more powerful
and have added myriads of improvements.

To interact with such a system, a so-called instruction set is provided, which
essentially is the language that the CPU understands. It is by use of this
predefined set of actions that the programmer can control the whole machinery
and ``tell'' the computer how to perform algorithms, read from and write to
memory, communicate with peripheral devices and so on.

Modern hardware-implemented computer architectures can execute instructions
extremely fast; the Tianhe-2 super computer performs more than
$33 \cdot 10^{15}$ floating point operations per second\cite{ieee-tianhe}! Being
this fast comes at the cost of expressiveness in the instruction set, which are
generally very low-level and specialized for a given CPU. This is where an
abstract machine becomes relevant.

\subsection{Abstract Machines}
\label{sec:background:abstract-machines}

Abstract machines are essentially computer architectures implemented in
software. They are developed and run just as regular programs which take some
input arguments and produce corresponding output, be it graphical interfaces,
textual output, network activity, etc.

To define what an abstract machine is, it helps to compare it with other similar
systems. There are many related technologies and concepts which have mixed
definitions and can stir up confusions. We present the most notable examples and
give our own broad definitions.

An \textbf{interpreter} is a program which reads high-level source code and
directly executes the code without translation into another language or native
code. Interpreters are easy to implement but tend to suffer from slower
performance\cite{circuitstoday}. Examples include Python\footnote{The Python
  Programming Language: \url{https://www.python.org}} and Ruby\footnote{The
  Ruby Programming Language: \url{https://www.ruby-lang.org}}.

A \textbf{compiler} translates source code from one language to another without
executing anything. The target language of compilers is typically an assembly
language understood by a CPU, regardless whether it is a hardware CPU or, as is
more relevant to this thesis, an abstract CPU implemented by an abstract
machine. A abstract machine is concerned with executing code where the compiler
translates it. Examples include GCC\cite{gnu:gcc} and GHC\footnote{The Glasgow
  Haskell Compiler: \url{https://www.haskell.org/ghc}}.

Further there is the concept of \textbf{Just-in-Time} (JIT) and
\textbf{Ahead-of-Time} (AOT) compilation which is often misunderstood as being
part of a compiler. JIT and AOT compilation is about \textit{when} compilation
takes place, not how. JIT compilers are implemented as part of abstract machines
or runtimes, and will compile an \term{intermediate representation} into native
code during run-time. For instance a language X may be compiled into language Y
which is understood by an abstract machine. Upon execution by the abstract
machine, language Y will be translated into native code and only then executed
on the hardware.

The term \textbf{virtual machine} is where some heavy debate can be sparked as
to how it relates to abstract machines. Some distinguish the terms by defining
an abstract machine to be a purely theoretical machine and virtual machine for
the corresponding implementation. Others define the virtual machine as being a
virtualization of an existing physical system and abstract machines as being a
purely software implemented machine. For the purpose of this thesis we regard
the term ``abstract machine'' and ``virtual machine'' to cover both.

Two overarching types of abstract machines exist, that serve two very different
purposes. First there is the \textit{system} abstract machine, which is a
program that emulates a full operating system, effectively running a copy of one
operating system in another. The program which executes the guest operating
system is called a HyperVisor which VirtualBox\footnote{\url{Oracle VirtualBox:
    https://www.virtualbox.org}} is a popular example of. The other type is the
\textit{process} abstract machine which runs like a regular program in the host
operating system and executes a single program. Typically when the execution of
the program stops, the abstract machine also exists. In this thesis we are
concerned solely with the process machine type.

\subsubsection{Components of an Abstract Machine}

The components of an abstract machine differ wildly with each implementation,
but there are common concepts that are present in most. We will briefly outline
those here, but will describe each in much more detail through
Section~\ref{sec:design} and Section~\ref{sec:implementation}.

The \textbf{instructions set} is as mentioned the language that the abstract CPU
understands. Typically each instruction has an opcode and a mnemonic name which
eases documentation, reading and writing assembly code for human beings. Further
some instructions can take arguments of varying types, e.g.~literal numbers or
strings, memory references, register names, stack indices, etc.

The \textbf{object model} defines the way in which the machine works with data,
and the memory organization of it. Classes, interfaces, methods and structures
are examples of such mechanism that aid the organization and development of
programs. We define an \textit{object} to be an entity comprised of a collection
of data fields of arbitrary kind, and potentially including methods which are
functions that are executed in the context of the data contained in the object.

Every computing system involves an \textbf{execution model} which describes how
code is executed during run-time. It includes a specification of how and where
arguments are passed to sub-routines, how values are returned from sub-routines,
how branching or jumping in the code is performed and similar operations.

\subsubsection{Stack and Register Based Implementations}
\label{sec:background:stack-vs-register}

In relation to the execution model there are generally two schools of
implementation strategy which dictates how code is written for the system.

One is the \textit{register} based implementation, in which data is handled
through registers. Values are moved between registers and instructions take
registers as arguments and typically also produce output to registers. This
model closely resembles common assembly code and thus the underlying hardware,
which can make it easier to optimize the code for execution on hardware. The
``Parrot VM'' (details in Section~\ref{sec:related-work:parrot}) is an example
of a register based abstract machine. The other model is the \textit{stack}
based implementation. As usual, implementation details differ widely from
project to project but generally it involves a single stack (per thread)
dedicated to operations on values, passing arguments, returning values and
everything else that the machine might support. Many instructions do not accept
any arguments but rather consume values from the stack which is expected to
contain ordered values of the correct type. The ``Java Virtual Machine''
(details in Section~\ref{sec:related-work:jvm}) is an example of a stack based
abstract machine.

\begin{figure}[h]
  \centering
  \begin{subfigure}[t]{.45\textwidth}
    \begin{lstlisting}[label={lst:background:stack}, caption=Stack based addition]
push 1
push 2
add
    \end{lstlisting}
  \end{subfigure}
  \begin{subfigure}[t]{.45\textwidth}
    \begin{lstlisting}[label={lst:background:register}, caption=Register based addition]
mov r1, $1
mov r2, $2
add r1, r2, r3
    \end{lstlisting}
  \end{subfigure}
\end{figure}

Listing~\ref{lst:background:stack} shows an example of simple addition using the
stack based model. After execution of the three instructions the value
\texttt{3} will be on top of the stack and the values \texttt{1} and \texttt{2}
will have been consumed by the \texttt{add}
instruction. Listing~\ref{lst:background:register} shows the same computation
using registers, where the result is the value \texttt{3} contained in register
3 (\texttt{r3}) and the two operands remaining in \texttt{r1} and \texttt{r2}.

There is no definitive answer to which is best, it is largely a matter of
preference and style. However, research done by rewriting the Java Virtual
Machine into a register machine has shown that register machines require 47\%
less executed instructions, is 32\% faster at executing programs, but requires
25\% more code\cite{shi05}. This is probably not the general case, but it gives
an indication of the pros and cons of each.

\subsection{Advantages and Disadvantages}

As with most things, there are both strengths and weaknesses of using abstract
machines. One of the most powerful aspects is that they are by definition
platform independent; programs written for the abstract machine can run on any
platform for which an implementation of the machine exists. Given that the
machine is written in a widely supported language, such as C or C++, it means
the machine can be compiled for use on a lot of platforms. Even embedded systems
can be supported by porting the machine, and potentially by exposing only a
subset of the full instruction set archtitecture (ISA).

Another compelling thing, arguably the main advantage, is that the instruction
set provided by the abstract machine is much more expressive than regular
assembly code. The details of hardware is abstracted away from the programmer
(or compiler), and a more high-level and easy-to-use ISA is provided. However,
some abstract machines are or were designed for a specific language or language
paradigm, resulting in a system which is hard to use as a code generation target
for other types of languages. An example is the Java Virtual Machine in which
the underlying tools and mechanisms of the machine are not exposed, making it
difficult to fit languages outside the original paradigm into the machinery.

It is easy to assume that abstract machines must suffer performance-wise because
of the added abstraction layer, but interestingly that is not necessarily the
case. With an added JIT compiler the performance may be as good or even better
than languages that are pre-compiled to native code\cite{mangione98,
  qwertie11}. One reason for this is that very aggressive optimization
algorithms can be implemented for the respective ISA, giving very effective
performance improvements for free regardless which platform the code will be
executed on.

\subsection{Existing Implementations}

There are scores of abstract machine implementations available, ranging from
small research projects to widely used industrial strength systems. Some of
today's most popular languages run on abstract machines, the two most well-known
being Java and C\#\cite{langpop}. We have selected some additional examples to
discuss due to their influence or otherwise interesting nature.

\subsubsection{The Java Virtual Machine}

The Java Virtual Machine (JVM) executes programs written in the Java Bytecode
format and is available on a huge amount of platforms; everything from TVs
through cell phones and media players to desktop PCs and Internet
browsers\cite{aboutjava}. JVM was developed specifically for the Java
programming language, hence the name, but due to its popularity a large amount
of other languages now target the JVM as their main run-time.

The JVM is a stack based machine and is heavily tied to the object oriented
language paradigm. Most of the low-level mechanism are inaccesible through the
ISA, which forces language implementors to fit everything into the Java object
model, which is inherently \term{Object-Oriented} and statically typed. As a
result it is not a straightforward task to implement dynamically typed or
functional languages on the JVM. Efforts have been made to asses this, most
notably the introduction of the \texttt{invokedynamic} instruction through the
Da Vinci project (see Section~\ref{sec:related-work:jvm} for details).

\subsubsection{Common Language Runtime}

Microsoft's Common Language Runtime (CLR) is the abstract machine that executes
the Microsoft Common Intermediate Language (CIL). Like the JVM it is stack based
and widely used on Microsoft Windows devices. An open-source port called
Mono\footnote{The Mono Project: \url{http://www.mono-project.com}} is developed
maintained for Mac OSX and Linux machines.

Like the JVM, the CLR is Object-Oriented by design and consequently suffers from
the same flexibility issues.
% TODO: more!

\subsubsection{Turing Machines}

Arguably the most fundamental and influential theorical (more precisely
mathematical) abstract machines are Turing machines. First described by Alan
Turing in 1937\cite{sep-turing-machine}, they were devised to help describe what
it means for an arbitrary task to be computable. They are essentially simple
state machines which operate on a (fictional and infinite) tape of cells of
value 0 or 1, and a read-write head that can either read or write the value of
the cell currently being scanned\cite{sep-turing-machine}. The machine acts
according to transition rules which describe what action will be performed,
given a machine state and current cell value. The action can be either a read or
write or a movement of the tape to the right or the left.

Given these simple mechanisms it is possible to perform any imaginable
computation however complex. Because of this, systems that can do so are called
Turing complete systems.

%TODO
%% SECD Machine
%% Examples of how functional code is translated to an object model (such as F# on .NET)

%%% Local Variables:
%%% mode: latex
%%% TeX-master: "../report"
%%% End:
