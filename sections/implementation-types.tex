The binary file, containing the program to be run, declares a type table for the
custom types created by the program. Through out the binary file, these types
are referenced through the type's index in this table. This table is static, so
the machine can analyze how many types it contains. The machine has it's own
type table, which is a statically sized array. The size is the sum of built-in
types and types from the binary file.This is called the machines \term{type
  table}. It holds references to {\it all} types used in the program.

The machines built-in types are described by a type tag, implemented as an enum:
\begin{ccode}
enum type_tag_e {
    ACTIVATION_ELEMENT = 0,
    ANY                = 1,
    BOOL               = 2,
    INT8               = 3,
    UINT8              = 4,
    ...
};
typedef enum type_tag_e type_tag;
\end{ccode}
The enums integer value is used to describe its index in the type table. There
will always only be {\it one} instance of each type, so two equal types can
never be referenced by two different pointer. This invariant lets the machine
check type equality through comparing the type pointers.

More specifically, all types in the machine is stored internally through
structs. Simple types implement the {\tt type\_base} struct, which is extended
in composite types. For instance, a reference type would be represented as:
\begin{ccode}
struct type_base {
    type_tag tag;
    int size;
};

struct type_ref {
    struct type_base base;
    struct type_base *ptr_type;
};

typedef type_base type;
\end{ccode}

All types are always passed around in the machine through the {\tt type}
name. If the type is for instance a reference, i.e. implementing the {\tt
  type\_ref}, is is cast so its reference specific attributes can be
referenced. The type of type is checked through the tag, defined in {\tt
  type\_base}. Another important entry in the type structure is the size
attribute. This is essential, as it tells the machine how much memory is needed
to store its value on either the stack or heap.

Types from the binary file are mapped from its index in the binary type table to
the machines type table. The types from the binary file is parsed lazily,
i.e. the first time the program tries to look-up a type through its index in the
binary type table, it is parsed and stored in the machine's type table. This is
done in the virtual machine implementation's {\tt vm\_lookup\_elf\_type}
function, which takes a reference to the type table and a type index in the
binary file.

To get a reference to a built-in type, one can look-it up through a look-up
function, which takes the name of the built-in type and the type table:
\begin{ccode} % TODO: update
type *lookup_type(type **ttable, type_tag tag)
{
    int i;
    for (i = 0; i < ttable_size; i++) {
        if (ttable[i] == NULL)
            continue;

        if (ttable[i]->tag == tag)
            return ttable[i];
    }

    log_errf(TYPE_ERROR, "could not find type with tag %d", tag);

    return NULL;
}
\end{ccode}
The function iterates the types in the type table, matching on its tag. The null
check is used to avoid types from the binary file not yet parsed, to be skipped.

Currently, if a type is not found, an error is thrown, halting the machine (TODO
exceptions).

Composite types are made through (TODO).

Most types can be converted, with some exceptions (TODO signature type). If a
value is converted to a type with smaller size, for instance an {\tt Int32} to
{\tt Int8}, the value is truncated (TODO exception)?

All types are garbage collected. This means, when a type is no longer used by
the program, it's memory is freed. (TODO)
