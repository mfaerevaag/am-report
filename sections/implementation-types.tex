%%% Local Variables:
%%% mode: latex
%%% TeX-master: "../report"
%%% End:

It is not only for stack or heap elements which we need to store information
about their type. Sub-routines and object fields also has type information which
we need to keep track of, but the information is not the same. For instance, a
stack or heap element has types describing the type of its value. A sub-routing
will need have a type signature, describing the types of it's parameters, a name
and possible properties. Similarly an object field will have (TODO).

Therefore, we need to abstract our type information a level higher, letting us
better describe different types of types. Instead of storing the different data
different places, we will have a centralized table, containing all meta
information. This will be named the \term{meta table} and will need to be
dynamically sized to support reflection and generating new types at run-time.

All meta objects in the machine will have a tag defining which of type of
information it contains.
\begin{ccode}
enum meta_tag_e {
    TYPE,
    METADATA,
    FIELD,
};
typedef enum meta_tag_e meta_tag;

struct meta_base_s {
    meta_tag tag;
};
typedef struct meta_base_s meta;

struct meta_type_s {
    meta base;

    type *type;
};
typedef struct meta_type_s meta_type;
\end{ccode}
In the case of {\tt TYPE} it will be cast to the {\tt meta\_type} which allows
access to the {\tt type} referenced described below.

We will describe each type of meta information in turn.

\subsubsection{Types}
The executable file, containing the program to be run, declares a type table for
the custom types created by the program. Through out the executable, these types
are referenced through the type's index in this table. This table is static, so
the machine can analyze how many types it contains. The types will be parsed
from the executable and added to the global meta table lazily. This reduces
the initial overhead of parsing the executable.

The machines built-in types are described by a type tag, implemented as an enum:
\begin{ccode}
enum type_tag_e {
    ACTIVATION_ELEMENT = 0,
    ANY                = 1,
    BOOL               = 2,
    INT8               = 3,
    UINT8              = 4,
    ...
};
typedef enum type_tag_e type_tag;
\end{ccode}
The enums integer value is used to describe its index in the meta table. There
will always only be {\it one} instance of each type, so two equal types can
never be referenced by two different pointers. This invariant lets the machine
check type equality through comparing the type pointers.

More specifically, all types in the machine is stored internally through
structs. Simple types implement the {\tt type\_base} struct, which is extended
in composite types. For instance, a reference is represented as:
\begin{ccode}
struct type_base_s {
    type_tag tag;
    int size;
};
typedef struct type_base_s type;

struct type_ref_s {
    type base;
    type *ref_type;
};
typedef struct type_ref_s type_ref;
\end{ccode}

All types are always passed around in the machine through the {\tt type}
name. If the type is for instance a reference, i.e. implementing the {\tt
  type\_ref}, it is cast so its reference specific attributes can be
referenced. The type of type is checked through the tag, defined in {\tt
  type\_base}. Another important entry in the type structure is the size
attribute. This is essential, as it tells the machine how much memory is needed
to store its value on either the stack or heap.

Types from the executable are mapped from its index in the binary type table to
the machines meta table. The types from the binary file is parsed lazily,
i.e. the first time the program tries to look-up a type through its index in the
binary type table, it is parsed and stored in the machine's type table. This is
done in the virtual machine implementation's {\tt vm\_lookup\_elf\_type}
function, which takes a reference to the type table and a type index in the
binary file.

TODO: update look-up shit
To get a reference to a built-in type, one can look-it up through a look-up
function, which takes the name of the built-in type and the type table:
\begin{ccode} % TODO: update
type *lookup_type(type **ttable, type_tag tag)
{
    int i;
    for (i = 0; i < ttable_size; i++) {
        if (ttable[i] == NULL)
            continue;

        if (ttable[i]->tag == tag)
            return ttable[i];
    }

    log_errf(TYPE_ERROR, "could not find type with tag %d", tag);

    return NULL;
}
\end{ccode}
The function iterates the types in the type table, matching on its tag. The null
check is used to avoid types from the binary file not yet parsed, to be skipped.

Currently, if a type is not found, an error is thrown, halting the machine (TODO
exceptions).

When trying to box a stack element trough the {\tt box} instruction, the top
element is popped off the stack and stored on the heap. An element with a
reference to the heap object is in turn pushed to the stack. The type of the new
element will be a {\tt Reference<t>} type, where {\tt t} is the type of the
stack element which was boxed. If this type is not found in the type table
already it will have to be generated. This is done by TODO.

Composite types are made through (TODO).

Most types can be converted, with some exceptions (TODO signature type). If a
value is converted to a type with smaller size, for instance an {\tt Int32} to
{\tt Int8}, the value is truncated (TODO exception)?

All types are garbage collected. This means, when a type is no longer used by
the program, it's memory is freed. (TODO)
