% Test-Driven Development (not strictly)
% unit tests
% black box testing
% def. terms

Throughout the project we have made heavy use of testing to ensure the
correctness of the implementation, i.e. whether it strictly follows the machine
specification~\ref{sec:spec}. We have both made use of \term{unit}- and
\term{integration-tests}. Here unit-tests have the purpose of testing a single
isolated component, typically a function. Here we mock a specific state of the
machine, or the state of a component in the machine, and then run isolated
functions where we {\it know} what the result should be, for which we can easily
test. By comparison, the integration-tests have a much larger scope. Their
purpose is to test the whole program without any regard to any specific
component. This effectively ensures each component is wired together correctly.

For parts of the implementation we have made use of \term{test-driven
  development} (TDD). Here we write the tests prior to implementing the actual
functionality being tested. This eliminates the danger of false positives, where
tests pass when they are not supposed to. It also offers an efficient work flow,
where the minimum functionality is implemented to make the tests pass.

\subsubsection{Unit-tests}
Unit-testing in C is a fairly simple process and essentially does not require
any framework or library. While not required, a simple set of \term{macros} will
greatly improve the readability and ease of writing tests.

% cmocka
In spite of this, we have chosen to use a unit-testing library called {\it
  cmocka}~\cite{cmocka}. It is a well tested and documented library, also used
by large projects like {\it libssh}~\footnote{Implementation of the SSH protocol:
  \url{http://libssh.org/}}. It offers several features which makes unit-testing
more powerful and simpler to write. For instance, it offers test suites and
\term{mocking} of objects which enabled us to set up a specific state of the
machine. With this state, we can run isolated tests, manipulating the state and
thereafter checking that the correct transformations and output has occurred.

Cmocka also allows the test program to recover from signaled exceptions,
e.g. {\tt SIGSEGV}, {\tt SIGILL}, etc. If a test in the test program triggers a
segmentation fault exception, for instance, it will not exit, but rather show
where the exception occurred and print useful debugging information like the
call stack.

Lastly, the library works on a wide range of platform and only requires the use
of the C library. This makes it possible to use the library on embedded
platforms and with different compilers.

% example
As an example, we will describe how we test parts of the stack implementation.

Firstly, we create a mocked state of a stack which we can use for our tests.
\begin{lstlisting}[language={[ANSI]C},caption={Unit-test setup procedure}]
  static int setup(void **state)
  {
    stack_t *s = malloc(sizeof(stack_t));
    stack_init(s, 100);

    assert_non_null(s->elements);
    assert_int_equal(100, s->max_size);
    assert_int_equal(0, sum_stack(s));

    *state = s;

    return 0;
  }
\end{lstlisting}

In the above code listing, we initialize a stack object, do some
simple assertions, and store it in the {\tt state} variable passed
with the setup function. As we see below, this state is given as
argument to all test cases which easily allows us to retrieve it by
dereferencing.
\begin{lstlisting}[language={[ANSI]C},caption={Unit-test of {\tt stack\_pop}}]
  static void test_pop(void **state)
  {
    stack_t *s = *state;

    uint32_t sum = sum_stack(s);

    stack_push(s, make_se_int(1));
    stack_push(s, make_se_int(2));
    stack_push(s, make_se_int(3));

    assert_int_equal(3, SE_INT(stack_pop(s)));
    assert_int_equal(2, SE_INT(stack_pop(s)));
    assert_int_equal(1, SE_INT(stack_pop(s)));

    assert_int_equal(sum, sum_stack(s));
  }
\end{lstlisting}

After having retrieved the stack for the {\tt state} parameter, we create an XOR
sum of the stack. When having tested the specific function, we assert that the
new XOR sum is the same as before, making sure that there is no unexpected
changes to the stack. Due to the simple nature of our {\tt stack\_sum} function,
this is not a guarantee, as different states of the stack may compute the same
sum. % TODO: more on stack sum?
In the body of our test case, the {\tt stack\_pop} function in this case, we
push several elements to the stack and assert that they are popped off again in
the correct order.

After all test cases has completed, a teardown function is run, reversing that
of the setup function. In the case of the stack tests, it simply frees the
memory allocated to the state of the stack.
% \begin{lstlisting}[language={[ANSI]C},caption={Unit-test teardown procedure}]
%   static int teardown(void **state)
%   {
%   stack_t *s = *state;
%   stack_free(s);
%
%   return 0;
% }
% \end{lstlisting}


\subsubsection{Integration-tests}
We want our integration tests to test the implementation in the fashion of
black-box testing, i.e. given some input, the program should produce a specific
output, regardless of how the program works internally. The easiest way of
accomplishing this is by taking the actual machine executable and running it
with certain programs and arguments. We have therefor created several different
programs, from which we know the desired output.

As an example we can check that the machine fails when stack underflow occurs,
i.e. popping more values off the stack then there are actual values. A program
doing exactly this could be:
% TODO: update with actual
\begin{lstlisting}[language={bytecode},caption={Machine program producing
    stack underflow}]
  fn main
  {
    pop
  }
\end{lstlisting}

The above machine program will try to pop an element off the stack without there
being any elements, producing a stack underflow error. We can test the output of
the machine by both asserting the exit code of the process, where we will have
specified the meanings of different exit codes, and by asserting what is written
to standard out and error ({\tt stdout} and {\tt stderr}). In the case of stack
underflow the exit code should be 4 and the output should match `underflow`
(TODO: update).

To automate this process we use shell scripting. For convenience, we have chosen
to utilize a shell testing framework (that's right), called {\tt
  shUnit2}~\footnote{shUnit2, unit testing for shell scripts:
  \url{http://code.google.com/p/shunit2/}}. This lets us streamline the
integration testing process, in contrast of manually running each machine
program and checking the output.

shUnit2 enables us to have an alias for the actual machine binary, letting us
set the target binary through our build system and piping standard error to
standard out. In our test cases we can then, by using the alias, store the
output and exit code in variables and do assertions based on their expected
values.
% TODO: update return code
\begin{lstlisting}[language={sh},caption={shUnit2 underflow test case}]
  test_stack_underflow()
  {
    output=$(am --file ${PDIR}/underflow.amb)
    ret=$?

    assertEquals "return code should be 4" 4 $ret
    assertTrue "output should include underflow" \
    "contains \"$output\" \"underflow\""
  }
\end{lstlisting}

In the above test case we run the machine with the underflow program, asserting
that the process exits with the correct code and that the output contains the
word `underflow`.
