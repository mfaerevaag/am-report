Throughout the project we have made heavy use of testing to ensure the
correctness of the implementation, i.e.~whether it strictly follows the machine
specification. We have both made use of unit- and integration-tests. Here
unit-tests have the purpose of testing a single isolated component, typically a
single function. We mock a specific state of the machine, or the state of a
component in the machine, and then run isolated functions where we {\it know}
what the result should be, which can easily be asserted. By comparison, the
integration-tests have a much larger scope, in that their purpose is to test the
whole machine without any regard to any specific component. This effectively
ensures each component is wired together correctly.

\subsubsection{Unit-Tests}

Unit-testing in C is a fairly simple process and essentially does not require
any framework or library. While not required, a simple set of macros will
greatly improve the readability and ease of writing tests.

% cmocka
In spite of this, we have chosen to use a unit-testing library called {\it
  Cmocka}\cite{cmocka}. It is a well tested and documented library, also used by
large projects like {\it libssh}\footnote{Implementation of the SSH protocol:
  \url{http://libssh.org}}. It offers several features which makes unit-testing
more powerful and simpler to write. For instance, it offers test suites and
mocking of objects which enabled us to set up a specific state of the
machine. With this state, we can run isolated tests, manipulating the state and
thereafter asserting that the correct transformations and output has occurred.

Cmocka also allows the test program to recover from signaled exceptions,
e.g. {\tt SIGSEGV}, {\tt SIGKILL}, etc. If a test in the test program triggers a
segmentation fault exception, for instance, it will not exit, but rather show
where the exception occurred and print useful debugging information like the
call stack.

Lastly, the library works on a wide range of platforms and only depends on the C
standard library. This makes it possible to use the library on embedded
platforms and with different compilers.

% example
As an example, we will describe how we test parts of the stack implementation.

Firstly, we create a mocked state of a stack which we can use for our tests.

%XXX assertions shouldn't happen in setup
\begin{lstlisting}[language={[ANSI]C},caption={Unit-test setup procedure}]
  static int setup(void **state)
  {
    stack_t *s = malloc(sizeof(stack_t));
    stack_init(s, 100);

    assert_non_null(s->elements);
    assert_int_equal(100, s->max_size);
    assert_int_equal(0, sum_stack(s));

    *state = s;

    return 0;
  }
\end{lstlisting}

In the above code listing, we initialize a stack object, do some simple
assertions, and store it in the {\tt state} variable passed with the setup
function. As we see below, this state is given as argument to all test cases
which easily allows us to retrieve it by dereferencing.

\begin{lstlisting}[language={[ANSI]C},caption={Unit-test of {\tt stack\_pop}}]
  static void test_pop(void **state)
  {
    stack_t *s = *state;

    uint32_t sum = sum_stack(s);

    stack_push(s, make_se_int(1));
    stack_push(s, make_se_int(2));
    stack_push(s, make_se_int(3));

    assert_int_equal(3, SE_INT(stack_pop(s)));
    assert_int_equal(2, SE_INT(stack_pop(s)));
    assert_int_equal(1, SE_INT(stack_pop(s)));

    assert_int_equal(sum, sum_stack(s));
  }
\end{lstlisting}

After having retrieved the stack for the {\tt state} parameter, we create an XOR
sum of the stack. When having tested the specific function, we assert that the
new XOR sum is the same as before, making sure that there is no unexpected
changes to the stack. Due to the simple nature of our {\tt stack\_sum} function,
this is not a guarantee, as different states of the stack may compute the same
sum, but will however catch most such problems.

In the body of a test case, the {\tt stack\_pop} function in this case, we push
several elements onto the stack and assert that they are popped off again in the
correct order.

After all test cases has completed, a teardown function is run, reversing that
of the setup function. In the case of the stack tests, it simply frees the
memory allocated to the state of the stack.

\subsubsection{Integration-Tests}

We want our integration tests to test the implementation in the fashion of
black-box testing, i.e.~given some input, the program should produce a specific
output, regardless of how the program works internally. The easiest way of
accomplishing this is by taking the actual machine executable and running it
with certain programs and arguments. We have therefor created several different
programs, from which we know the desired output.

As an example we can check that the machine fails when stack underflow occurs,
i.e.\ popping more values off the stack then there are actual values.

A valid test program for this would consist of a single instruction, \instr{POP,
  1}, producing a stack underflow error. We can test the output of the machine
by both asserting the exit code of the process, where we will have specified the
meanings of different exit codes, and by asserting what is written to standard
out and error ({\tt stdout} and {\tt stderr}). In the case of stack underflow
the exit code should be 4 and the output should match ``underflow'' (TODO:
update).

To automate this process we use shell scripting. For convenience, we have chosen
to utilize a shell testing framework, called shUnit2\footnote{shUnit2, unit
  testing for shell scripts: \url{http://code.google.com/p/shunit2/}}. This lets
us streamline the integration testing process, in contrast to manually running
each machine program and checking the output.

shUnit2 enables us to have an alias for the actual machine binary, letting us
set the target binary through our build system and piping standard error to
standard out. In our test cases we can then, by using the alias, store the
output and exit code in variables and do assertions based on their expected
values.

\begin{lstlisting}[language={sh},caption={shUnit2 underflow test case}]
test_stack_underflow()
{
  output=$(am --file ${PDIR}/underflow.amb)
  ret=$?

  assertEquals "return code should be 4" 4 $ret
  assertTrue "output should include underflow" \
  "contains \"$output\" \"underflow\""
}
\end{lstlisting}

In the above test case we run the machine with the underflow program, asserting
that the process exits with the correct code and that the output contains the
word `underflow`.

%%% Local Variables:
%%% mode: latex
%%% TeX-master: "../report"
%%% End:
