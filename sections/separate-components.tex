A complete abstract machine implementation is not only made up of the system
that interprets byte code. It can include various components that either work
completely isolated or interfaces with the byte code interpreter. Exactly what
other components are part of the system varies widely among machines. In the
following we will describe three such components that we have either made a
phony implementation of or which would have been an essential addition to the
system.

\subsection{Garbage Collector}
\label{sec:separate-components:gc}

Heap objects to which there are no references, and so are unreachable, are known
as garbage. Such regions of memory can never be used again and have no useful
purpose, hence the name. In languages that manually manage memory (e.g. C), such
objects are regarded as a \textit{leak}, because the memory they inhibit can
never be reclaimed and used for other things, but will continue to take up
memory until the program exits. The garbage collector's responsibility is to
\textit{automatically} keep track of which heap objects are in use, and which
can be safely freed, relieving the programmer of the task of manually freeing
allocations.

There are generally two types of garbage collectors: reference counters and
tracing garbage collectors. Reference counting simply keeps track of how many
references there are to a given object and when that count becomes zero, the
object is regarded as garbage and will be freed. This is an efficient scheme for
simple data structures but there is a critical issue with circular
structures\cite{bruno}. Imagine a singly-linked circular linked list which is
accessible only via some variable named \code{head}. If \code{head} becomes
null, then all members of the list will still have a reference to them because
the list is circular, only there is no way to access them. Thus the reference
counter cannot determine that the list is garbage and a leak has occured.

Tracing garbage collectors work by maintaining a graph representation of
allocated objects that it can use to infer which objects are garbage and which
are still in use. This is known as mark-and-sweep collection because the
collector, upon collection, traverses the graph and marks objects that are still
in use. It then sweeps the heap and reclaims the resources used by the objects
that were not marked as being in use.

\thename{} uses a garbage collector through the interface defined in
\code{gc.h}, which essentially exposes four functions:

\begin{description}
\item[allocate] Allocates memory for a new heap object
\item[release] Releases a heap object for tracking by garbage collection,
  essentially marking the object ``ready'' to be tracked. This does \textit{not}
  release the memory used by the object.
\item[de\_ref] Tells the garbage collector that a dereference of the object has
  occured, indicating that it is use.
\item[re\_ref] Tells the garbage collector that an object is no longer
  referenced.
\end{description}

We have made a phony implementation of that interface using \code{malloc} for
allocation and left the rest of the functions unimplemented, which basically
means that all heap allocations are leaked. This is not a big issue when
developing the machine, because we do not need to run memory heavy programs, but
for any actual use it is obviously critical that the garbage collector works. As
with the threading, the implementation is wrapped in auxiliary function to
simplify the process of changing the actual implementation without changing the
interface.

The biggest advantage of using a garbage collector is that it prevents most
memory leaks and can significantly simplify development. It does however come
with a set of disadvantages, most prominently a performance overhead. Depending
on the type of garbage collector used, it is not known when and how often the
collector kicks in and frees memory, which can result in undesirable hang-ups
during run-time.

\subsection{Byte Code Verifier}

The job of the byte code verifier is to, as far as possible, prevent run-time
errors by detecting illegal or invalid byte codes \emph{statically}, that is,
before the code is executed by the interpreter. The verifier is a completely
separate system and the interpreter itself does not need to interface with it at
any time, nor is verifier aware of the interpreter. It is however usually
implemented as a tier above the interpreter in the complete machine meaning that
code must pass the verifier before it is allowed to be executed.

The byte code verifier is separated from the system that generates the byte
code. A compiler is responsible for verifying the semantics and syntax of a
particular language, which is something the machine and byte code verifier is
obviously unaware of.

In theory the interpreter could assume that any code fed to it is valid, but in
practice this is not viable. There are issues that the verifier cannot or might
not detect, such as certain type errors (see below) and issues related to
dynamic language features. In addition there are some things that we
intentionally re-verify during run-time because of security critical
reasons. For instance, we check that stack operations never reference the
activation element of sub-routines.

There are a lot of things that the verifier can or must check to ensure code
validity. Some are specific to abstract machine specifications and
implementations, but some are common inspections that apply to most kinds of
byte code, or indeed code in general. We have selected some that are either
specific to \thename{} or general-purpose verifications that would be suitable
for a verifier for \thename{}, and give a brief description of each. Some of the
below are inspired by the Java Virtual Machine's specification of byte code
verifier\cite{jvm-spec}.

\subsubsection{Instructions Prefixes and Arguments}

Section 8.4 of \thename{} specification constrains the order and composability
of some op-code prefixes. For instance the \code{unsafe} prefix must directly
proceed the instruction op-code.

Arguments to instructions can be verified in terms of size, type and
existence. Instructions take an argument, \emph{must} do so because the
interpreter will parse the next byte(s) as argument. It can be difficult to tell
where an argument is missing because an op-code is a valid numeric value, but at
some point (most likely in the end of a sub-routine) it will be possible to
detect an unbalanced ratio of arguments to op-codes. For instance if the
\code{large} prefix is used then the argument(s) must be four bytes long.

\subsubsection{Stack Verification}

The stack can be verified to never reference illegal elements, such as the
activation element. Neither is it allowed to reference or modify elements
outside the range of the current sub-routine.

Because stacks are unbounded, stack overflows are not possible, but stack
underflows can occur if an instruction attempts to pop more elements than are
currently on the stack.

\subsubsection{Branching and Sub-routines}
\label{sec:separate-components:verifier:branch}

Branching instructions take an arbitrary offset argument, which must be verified
to not lie outside the bounds of the current sub-routine. Further, each
\code{call} and \code{invoke} instruction must have a corresponding
\code{return} instruction so as not to accumulate activation elements on the
stack.

\subsubsection{Data-Flow Analysis}

One of the most common programming errors is to attempt to dereference an
uninitialized or null-value reference. Another is to use an index of an
statically sized array which is outside the bounds of the array. Analysis of the
flow of data in a program can reveal many such errors, usually by constructing a
so-called Control-Flow Graph, which represents what code paths might be taken
during execution.

\subsubsection{Type Checking}

Type checking is a complex task that attempts to verify that the types of
parameters to instructions and sub-routines match the ones provided as
argument. Statically typed languages are superior in this context because all
variables must either have a type annotation or the type must be
inferrable. Dynamically typed languages can make no compile-time guarantees for
type safety, exactly because the types are determined and sometimes constructed
at run-time. The verifier must thus do ``what it can'', and remaining mismatched
types will result in run-time errors.

\subsection{Assembler and Linker}

%TODO remove first paragraph? at least don't mention that svenka didnt finish his sheit
The assembler and linker will be implemented through porting (TODO: define?)
parts of the GNU Toolchain. The necessary components are the GNU
Assembler\cite{gnu:as}, {\bf as}, commonly known as GAS, and the GNU Linker,
{\bf ld}\cite{gnu:ld}. These ports are not yet finished, and are therefore not
yet used in the machine.

A linker combines several compiled object files into a single binary file,
executable by \thename{}. This often occurs as the last step of the compilation
process, where each file are compiled separately and finally linked together
with a linker. {\bf ld} is the standard linker for GCC, TODO.

The GNU Assembler is actually a family of assemblers, supporting many
front-ends, or languages. Many other assemblers need to pass over the source
code multiple times to be able to produce a valid output, while {\bf as} was
designed to do assemble in a single pass. This makes for a more complex assembly
process, but in turn removes the overhead of passing over the source code
multiple times.

The reason for choosing to port new implementation of both of these components
is to make use of their arguably, good performance and
configurability\cite{NEEDED}. Creating something new from scratch would have
taken a significant amount of time to be able to create something that could
even compare in performance.

% TODO: more

\subsection{Byte Code Optimizer}

Compilers that generates byte code will almost always do heavy optimizations of
the generated byte code, but optimization can also be a part of the machine
itself. It can either be an isolated component sitting in between the verifier
and the interpreter or it can be integrated in the machine where it optimizes
code when it is found to be suitable. When integrated in the machine, code that
is executed repeatedly (known as a Hot-Spot) can be further optimized because
even small optimizations in a Hot-Spot can result in great overall run-time
improvements.

Code optimization is a whole research area in itself, and so there are
multitudes of algorithms for detecting and optimizing certain constructs of
code. We have again selected a few that we believe are fundamental.

It is possible to write code that can never be reached in any execution path,
known as dead code. A simple example is any code following a return instruction
which will be dead because the return instruction jumps away. A more complex
example (in terms of detection at least) is presented in
Listing~\ref{lst:separate-components:dead-code} in high-level pseudo code for
readability. Dead code is redundant and can safely be removed from the program,
a process known as \textbf{dead code elimination}.

\begin{ccode}[
  caption={Example of dead code},
  label={lst:separate-components:dead-code}]
if (x) {
  statement 1
}
else if (!x) {
  statement 2
}
else {
  // This block is dead code since one of the above conditionals will always be true
  statement 3
}
\end{ccode}

Sub-routine calls carry significant overhead, especially when small code blocks
are frequently executed. This can be mitigated by \textbf{inlining} sub-routine
code directly into the call-site, thus preventing the function call, but usually
resulting in more lines of code because the sub-routine's code is duplicated
whenever it is used.

Some expressions are so-called constant expressions, that is, they do not depend
on any user provided input and will always evaluate to the same result. Such
expressions can be calculated at compile-time, avoiding the run-time
computation. This is known as \textbf{constant
  folding}. Listing~\ref{lst:separate-components:inline-constant-folding} shows
and example of inlining and constant folding.

\begin{ccode}[
  caption={Sub-routine inlining and constant folding},
  label={lst:separate-components:inline-constant-folding}]
function double (x) { return x + x }

// Standard call to `double'
x = double(2);
y = double(x);

// `double' inlined
x = 2 + 2;
y = x + x;

// Constant folding
x = 4;
y = x + x;

// `x' can be inlined
y = 4 + 4

// Resulting in another case of constant folding
y = 8
\end{ccode}

A lot of so-called \textbf{peephole optimizations} can be done on small snippets
of code. They are generally very isolated cases of optimization that either
reduce the number of instructions or replace instructions with faster ones. For
instance \code{x * 2} can be replaced with \code{x << 2} since shifting is
much faster than multiplication.

Another trivial optimization is to remove unused sub-routines. If a sub-routine
is never called it is redundant and can be safely removed from the program,
saving space and file loading time.

Looping constructs can, among other transformations, be optimized by detecting
\textbf{loop-invariant code}, i.e.~code that is evaluated in each loop
iteration, but always evaluates to the same value. Such code can be moved
outside the body of the loop so as only to evaluate the code once.

%%% Local Variables:
%%% mode: latex
%%% TeX-master: "../report"
%%% End:
